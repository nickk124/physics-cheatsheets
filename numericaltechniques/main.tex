\documentclass[10pt]{article}
\usepackage[margin=0.4in]{geometry}
\usepackage{amsmath}
\usepackage{amssymb}
\usepackage{graphicx}
\usepackage{lmodern}
\usepackage{float}
\usepackage{physics}
\usepackage{minibox}
\newcommand*\diff{\mathop{}\!\mathrm{d}}
\newcommand\tab[1][0.5cm]{\hspace*{#1}}
\begin{document}
	\fontsize{6pt}{7pt}\selectfont
\section*{Number Representation and Root Finding/Fixed Point Methods}
\minibox[frame]{\textbf{Number Representation}
	\\Range: min and max vals that can be represented
	\\precision: min increment b/w adjacent vals
	\\Decimal Representation of 260.25: \textbf{1's place is $10^0/2^0$!}
	\\$2\times10^2+6\times10^1+0\times10^0+2\times10^{-1}+5\times10^{-2}$
	\\Binary Representation of 260.25: ($2^{10}=1024$)
	\\$2^8 + 2^2 + 2^{-2} = 100000100.01$
	\\Floating Point representation (scientific notation):
	\\d.$ddddd...\times10^p$
	\\$d$ is int [0,9]. behind dec. pt is mantissa, $p$ is exponent
	\\BINARY floating pt:
	\\$1.bbbbbbbbbb \times 2^{bbbb}$ (each $b$ is a bit (0 or 1).
	\\REMEMBER exponent is also in binary
	\\IEEE floating pt standard: $f=sMB^{e-E}$
	\\$s$ is sign bit, $M$ is mantissa and leading digit
	\\$B$ is base $e$ is exponent $E$ is bias
	\\\textbf{Error Sources in Programming}
	\\syntax errors
	\\runtime errors (code compiles but fails running)
	\\numerical errors
	\\physics errors
	\\can be systematic or uncontrolled
	\\rounding errors



}
\minibox[frame]{\textbf{Basic Root Finding}
	\\compute f(x) over mesh, look for sign changes
	\\\textbf{Bisection root finding}:
	\\0)plot and inspect visually
	\\1)Choose interval [a,b] s.t. soln exits to $f(x)=0$
	\\2)Calculuate first estimate as $(a+b)/2$
	\\3)Determine where soln is (to left or right of midpt.)
	\\4)select new brackets; stop when $\abs{a-b} <$ tolerance.
	\\\textbf{Secant root finding}:
	\\1)Choose two points $x_1$ and $x_2$
	\\2)Calculate $f_1$ and $f_2$
	\\3)Calculate fire estimate of soln by
	\\extrapolating to x axis with straight line, hitting $x_3$:
	\\$x_3=(f_2x_1-f_1x_2)/(f_2-f_1)$
	\\4) check tolerance
	\\5)update bracket with $x_2\rightarrow x_1$, $x_3\rightarrow x_2$
	\\works even if $x_1$ and $x_2$ don't bracket root
	\\converges if  $\abs{f_{n+1}} < \abs{f_n^\prime}\abs{x_{n+1}-x_n}$
	


}
\minibox[frame]{\textbf{Newton-Raphson Root Finding}
	\\for $f(x)=0$, write out $f^\prime(x)$
	\\choose some pt $x_0$ visibly close to root
	\\$x_{n+1} = x_n - f(x_n)/f^\prime (x_n)$
	\\\textbf{Orders of Convergence of Methods}
	\\$\varepsilon_n=x_{n+1}-x_n$ (also can be tolerance).
	\\$m$ is the actual order of conv.
	\\Bisection: $\varepsilon_{n+1}=\varepsilon/2$
	\\Secant: $\varepsilon_{n+1} \simeq \text{const}\times \varepsilon_n^{1.618}$
	\\N-R Method: $\varepsilon_{n+1} \simeq \text{const}\times \varepsilon_n^{2}$
	\\To determine $m$, test to see if $\varepsilon_{n+1}/\varepsilon_n^m$ is constant
	\\Will be constant for correct order $m$
	\\\textbf{General Fixed Pt Methods}
	\\Useful to design methods that converge
	\\Rewrite your needed $f(x)$ as $g(x) = x$
	\\Iteration: $x_{n+1} = g(x_n)$
	\\N-R is a type of F-P iteration
	\\\textbf{Convergence of FP methods}
	\\Converges if in neighborhood of FP, $\abs{g^\prime(x)} < 1$
	\\(Lipschitz continuous)
	\\So for N-R, converges if $\abs{g^\prime(x)} =\abs{\frac{f(x)f^{\prime\prime}(x)}{(f^\prime(x))^2}}< 1$

}
\section*{Multi-Dimensional Root Finding, Interpolation}
\minibox[frame]{for $n$ equations and $m$ unknowns:
	\\$m=n$:
	\\linear in unknowns: use linear alg interp.
	\\nonlinear in unknowns: multi-dim root finding
	\\$m<n$: overdetermined:
	\\linear in unknowns: can use lin alg or least squares
	\\nonlinear in unknowns: nonlinear least sq fitting
	\\\textbf{Multi Dim Root finding:}
	\\Consider system of form:
	\\$F_1(x_1,x_2,...,x_n)=0$
	\\$F_2(x_1,x_2,...,x_n)=0$
	\\$\vdots$
	\\$F_n(x_1,x_2,...,x_n)=0$
	\\Expand fns in Taylor series:
	\\$F_i(\vec{x}+\delta\vec{x})=F_i(\vec{x}) + \sum\limits_{j=1}^n\pdv{F_i}{x_j}\delta x_j+O(\delta x^2)$
	\\If close to root, we can approximate:
	\\$\vec{F}(\vec{x}+\delta\vec{x})=\vec{F}(\vec{x})+\boldsymbol{J}\cdot \delta\vec{x}=0$
	\\So $\boxed{\vec{x}_{n+1}=\vec{x}_n+\delta\vec{x}}$
	\\$\boldsymbol{J}\cdot\delta\vec{x}=-\vec{F}$
	\\$\boxed{\delta\vec{x}=-\boldsymbol{J}^{-1}\cdot\vec{F}}$
	\\(We're solving for $f_i =0$)
	\\$\boldsymbol{J}= \left[\pdv{\vec{f}}{x_1}\cdots \pdv{\vec{f}}{x_n}\right]$
	\\$\boldsymbol{J}= \begin{pmatrix} 
	\pdv{f_1}{x_1}&\cdots &\pdv{f_1}{x_n} \\ \vdots &\ddots &\vdots \\ \pdv{f_m}{x_1} & \cdots & \pdv{f_m}{x_n}\end{pmatrix}$
	\\$\vec{x}=(x_1\,x_2\,\cdots x_n)$
	\\\textbf{Multi-Dim Convergence}
	\\Contraction mapping: converges on a FP 
	\\after multiple iterations
	\\The surface of bracketing contracts 
	\\around FP after each iteration
	\\So, $G(x)$ is a contraction mapping (converges)
	\\to some $x_0$ if all eigenvals of $\boldsymbol{J}$
	\\are $<1$ in abs value
	\\\textbf{How to}:
	\\given some eqn with variables and unknown constants,
	\\plug in known vals for variables to get system of eqns
	\\to solve for consts, (do $f_i=0$)
	
}
\minibox[frame]{
\minibox[frame]{\textbf{Interpolation: regular polynomial}
	\\Motivation: functions may be smooth but tough to evaluate
	\\measurements
	\\predictions based upon complicated calculations
	\\Types: Polynomials
	\\Piece-wise polynomials (splines)
	\\Trig functions (Fourier series)
	\\\textbf{Polynomial Interpolation}
	\\Given $n$ points (x,y), they uniquely determine 
	\\a polynomial of degree $m = n-1$
	\\Advantages of using polynomial: smooth, 
	\\continuous, infinitely differentiable
	\\Disadvantages: oscillations get crazy if high degree
	\\Process: need to find $f(x)=a_mx^m+a_{m-1}x^{m-1}+\cdots + a_0$
	\\need to find $n=m+1$ coeffs:
	\\$(x_1,y_1)\quad a_mx_1^m+a_{m-1}x_1^{m-1}+\cdots+a_0=y_1$
	\\$(x_2,y_2)\quad a_mx_2^m+a_{m-1}x_2^{m-1}+\cdots+a_0=y_2$
	\\$\vdots$
	\\$(x_n,y_n)\quad a_mx_n^m+a_{m-1}x_n^{m-1}+\cdots+a_0=y_n$
	\\Lagrange polynomial interpolation:
	\\$f(x)=\sum\limits_{i=1}^ny_iL_i(x)\qquad L_i(x)=\prod\limits_{j=1;j\neq i}^{n}\frac{x-x_j}{x_i-x_j}$
	\\less efficient but more numerically stable
	\\can be applied to regular polynomial fitting
	\\and also splines
	
}
\\\minibox[frame]{\textbf{Interpolation: Splines}
	\\finding interpolating function for each pair of adjacent pts
	\\\textbf{linear}: $f(x)=y_2\frac{(x-x_1)}{(x_2-x_1)}+y_1\frac{(x-x_2)}{(x_1-x_2)}$
	\\continous, nondifferentiable
	\\\textbf{quadratic:} $f_i(x)=a_ix^2+b_ix+c_i$
	\\conditions: $f_i(x_{i+1})=y_{i+1}$, $f_i(x_i)=y_i$,
	\\$f^\prime_i(x_{i+1})=f^\prime_{i+1}(x_{i+1})$
	\\need one more: i.e., use linear interp for 1st interval
	\\continuous and differentiable
	\\\textbf{cubic:} $f_i(x)=a_ix^3+b_ix^2+c_ix+d_i$
	\\$4(n-1)$ coeffs total ($n$ points)
	\\conditions: match functions, 
	\\and match first and second derivatives for interior pts
	\\Summary: $f_n(x_n)=f_{n+1}(x_n)=y_x$, $f^\prime_n(x_n)=f_{n+1}^\prime(x_n)$,
	\\$f^{\prime\prime}_n(x_n)=f^{\prime\prime}_{n+1}(x_n)$
	\\take second derivative to be zero at endpoints
	
}
}
\minibox[frame]{
\minibox[frame]{\textbf{MISC}
\\Taylor: $\sum\limits_{n=0}^\infty\frac{f^{(n)}(x_0)}{n!}(x-x_0)^n$
}
\\\minibox[frame]{\textbf{Sparse Matrices}
	\\many more zeroes than not
	\\sparsity: no. zeros/ no. total elements
	\\Inverses usually exist
	\\Advantage: faster to compute
	\\Taylor approx Forward diff:
	\\$f^\prime (x)\simeq (f(x+h)-f(x))/h$ h is mesh
	\\Backward diff:
	\\$f^\prime (x)\simeq (f(x)-f(x-h))/h$
	\\Central diff:
	\\$f^\prime (x)\simeq (f(x+h)-f(x-h))/2h$
	\\h too large: truncation error
	\\h too small? rounding error for 2 very close vals
	\\From Taylor exp,
	\\$f^{\prime\prime}(x)\simeq (f(x+h)+f(x-h)-2f(x))/h^2$
	\\Matrix $D^{(2)} = 1/h^2\times M$
	\\$M$ has diagonals = -2, and elements left 
	\\and right of those 1
	\\all others are zero 
	\\(but other corners can be 1 if periodic)
	\\\textbf{Spline Example}
	\\Conditions: $f_i(x_i)=y_i$, $f_i(x_{i+1})=y_{i+1}$
	\\$f_i(x_i)=f_{i+1}(x_i)=y_n$
	\\$f_i^\prime(x_i)=f_{i-1}^\prime(x_i)$, $f_i^{\prime\prime}(x_i)=f_{i-1}^{\prime\prime}(x_i)$
	\\You know that $f_i(x)=ax^3+bx^2+cx+d$
	\\Can differentiate this as needed
}
}
\\
\minibox[frame]{
	\minibox[frame]{\textbf{Advantages of different types of root finding}
		\\N-R Pros: usually convergent for start points that are good enough
		\\Converges fast, only requires one starting pt
		\\N-R Cons: is derivative hard to do by hand, or may not exist
		\\can be divergent if slope not similar to that near root, such as for oscillatory, 
		\\which bisection can be better for
		\\Bisection cons: might not converge if there are poles or an odd number of roots inside bracket, or if not bracketed.
	}
	\\\minibox[frame]{\textbf{Bilinear Interpolation}
		\\make square grid of $\vec{x}$ that brackets chosen point
		\\numbered 1 through 4 CCW from lower left corner
		\\$t=(x-x_1)/(x_2-x_1)$, $u=(y-y_1)/(y_4-y_1)$
		\\$z_i\equiv f(\vec{x}_i)$
		\\$f(\vec{x})=(1-t)(1-u)z_1 + t(1-u)z_2 +tuz_3 + (1-t)uz_4$
		\\Continuous always, but not differentiable at support points (due to sharp discontinuities)
		
	}
}
\newpage
\section*{Least-Squares Fitting, Vector Spaces, Matrices, Linear Systems, Eigenproblems}
\minibox[frame]{\textbf{Least-Squares Fitting}
	\\More equations (n) than Unknowns (m)
	\\$f(x;a_0,a_1,\cdots, a_m)=y$
	\\Where $a_m$ are the $m+1$ free fit parameters
	\\In general, $x$, $y$ can be vectors of dimension $n+1$:
	\\this gives $n+1$ equations:
	\\$f(x_i;a_0,a_1,\cdots a_m)=y_i$
	\\Solve for free-fit parameters, minimize residuals:
	\\$S(a_0,a_1,\cdots,a_m) = \sum_{i=0}^{n}[y_i-f(x_i)]^2$
	\\\textbf{Many physical processes exhibit normal distribution:}
	\\Prob density: $\mathcal{P}(y) = \frac{1}{\sigma\sqrt{2\pi}}\exp{-(y-\mu)^2/2\sigma^2}$
	\\Mean: $\mu$, STDEV: $\sigma^2 \simeq \frac{1}{N-1}\sum_i(y_i-\mu)^2$
	\\Minimizing residuals gives the most probable parameters
	\\Least-sq fitting is a specific example of estimating max of likelihood function
	\\\textbf{Nonlinear Least-Squares Fitting}
	\\$f(x_i,\beta)=y_i$, $\beta$ are free-fit parameters
	\\Can use Newton-Raphson: $\boldsymbol{J}\cdot \delta\boldsymbol{\beta} = -\boldsymbol{F}$
	\\Levenberg-Marquardt (standard): used in python
	\\$(\boldsymbol{J}^T\boldsymbol{J}+\lambda\text{diag}(\boldsymbol{J}^T\boldsymbol{J}))\delta\boldsymbol{\beta}=-\boldsymbol{J}^T\boldsymbol{F}$
	\\$\lambda$ is damping factor
	\\need functional form to fit, starting data and parameters guesses

}
\minibox[frame]{\textbf{Vector Spaces, Vectors, Matrices}
	\\A set that is closed over finite vector addition and scalar multiplication
	\\Examples of vector spaces and their basis sets:
	\\Cartesian ($\hat{i},\hat{j},\hat{k}$)
	\\Polynomials ($1, x, x^2 \cdots$)
	\\Set of variables to root find over
	\\Scalars form a vector field
	\\Vector spaces defined over field
	\\\textbf{Inner Products/Dot Products}
	\\$\vec{v}\cdot\vec{w} = (\vec{w}\cdot\vec{v})^*$
	\\$a\vec{v}\cdot\vec{w} = a(\vec{v}\cdot\vec{w})$
	\\$(\vec{v}+\vec{u})\cdot\vec{w} = \vec{v}\cdot\vec{w} + \vec{u}\cdot\vec{w}$
	\\Two vectors orthogonal if inner product is zero
	\\polynomials like Legendre polys result from orthogonalization
	\\Linear operator: $A(\vec{x}+\vec{y}) = A\vec{x} + A\vec{y}$
	\\and scalar property
	\\and 2 real non-colinear vectors form a basis of $\mathbb{R}^2$
	\\\textbf{Dimension}: of space $V$, number of vectors in a basis
	\\\textbf{More on linear operators}
	\\always map $\vec{0}$ onto $\vec{0}$
	\\kernel: set of vectors $\vec{x}$ such that:
	\\$\ker(f) = \{\vec{x}\in V : f(\vec{x}) = 0\}$
	\\image of $f$: set of output vectors within output space $W$.
	\\$\dim(\ker(f)) + \dim(\text{im}(f)) = \dim(V)$
	\\\textbf{Inverse}: $M^{-1}Mv=v$
	\\Inverse exists if $\ker(M) = 0$, $\det(M) \neq 0$,
	\\rows of $M$ form set of linearly indep vecs
	\\$\det(AB)=\det(A)\det(B)$, $\det(kA) = k^n\det(A)$
	\\\textbf{Matrix Norms}
	\\Euclidean Norm: $\norm{A}\equiv \sqrt{\sum\limits_{i=1}^m\sum\limits_{j=1}^n a_{ij}^2}$
	\\$\norm{aM} = \abs{a}\norm{M}$, $\norm{A\vec{v}} \leq \norm{A}\norm{\vec{v}}$
	\\Triangle Inequality: $\norm{A+B}\leq \norm{A}+\norm{B}$
}
\\
\minibox[frame]{\textbf{Solving Linear Systems}
	\\\textbf{If det=0, matrix is lin depend}
	\\\textbf{Conditioning}: $\abs{\det(A)} / \norm{A}$ the larger the better
	\\Condition number $C_A:=\norm{A^{-1}}\cdot\norm{A}$
	\\$\frac{1}{C_A}\frac{\norm{\vec{r}}}{\norm{\vec{b}}}\leq\frac{\norm{\vec{e}}}{\norm{\vec{x}_{true}}}\leq C_A\frac{\norm{\vec{r}}}{\norm{\vec{b}}}$
	\\Useful for estimating magnitude of error
	\\Also, well conditioned if diagonal dominant (magnitude of each
	\\diagonal greater than sum of magnitudes of other row elements)
	\\residual $\vec{r} = A\vec{x} - \vec{b}$
	\\error $\vec{e} = \vec{x} - \vec{x}_{true}$
	\\$A\vec{x}=\vec{b}\qquad \vec{x} = A^{-1}\vec{b}$
	\\Lower triangular: forward substi:
	\\$x_m=\left(b_m - \sum\limits_{n}a_{mn}x_n\right)/a_{mm}$
	\\$m$ goes from first row to last row, the sum 
	\\sums up all of the previous iterations
	\\Upper triangular: backward substi:
	\\$m$ begins at lowest row. n iterates from $m+1$ to $1$ (not $0$).
	\\\textbf{Gaussian Elimination}
	\\\textbf{Steps} to get to upper triangular form:
	\\1) Use 1st row (pivot) to eliminate the first element 
	\\of each of the lower rows in sequence
	\\2) Use 2nd row (pivot) to elim 2nd element of each of the lower rows
	\\etc
	\\No. operations: $\sum\limits_{i=1}^{n-1}(2n-(2i-1))(n-i)$
	\\\textbf{DETAILED STEPS:}
	\\ 1) Take $R_1$, divide by $a_{00}$, multiply by $a_{10}$
	\\ 2) Subtract that from $R_2$
	\\ Do same to lower row:
	\\ 3) Use $R_1$ and divided by $a_{00}$, multiply by new $a_{20}$
	\\ etc...
	\\\textbf{Algorithm:} 
	\\\texttt{for i in range(0,n-1):} 
	\\\tab \texttt{for j in range(i+1,n):}
	\\\tab\tab \texttt{c = A[j,i]/A[i,i]}
	\\\tab\tab \texttt{A[j,i] = 0.0}
	\\\tab\tab \texttt{A[j,i+1:n] -= c*A[i,i+1:n]}
	\\\tab\tab \texttt{b[j] = b[j]-c*b[i]}
	\\\textbf{LU Decomposition}
	\\$A=LU$ L,U triangular matrices
	\\Find with GE: U is matrix from GE
	\\L is matrix of multipliers (\texttt{c}) in code used to
	\\eliminate each element, put in place where that element was.
	\\Advantage over GE: less computationally intensive.
	\\Furthermore with multiple b's and same A, you'll need 	
	\\to do full GE for every b, but with LU, only needed once.
	\\One GE is O($N^3$), one trigular solving is O($N^2$).
	\\LU only requires one of each of these for many b's
}
\minibox[frame]{
\minibox[frame]{\textbf{Eigenvalue Problems}
	\\$A\vec{w}=\lambda\vec{w}$
	\\$A\vec{x}=\vec{b}$ take form $A\vec{w}=\lambda\vec{w}$
	\\$(A-\lambda I)\vec{w}=\vec{0} \qquad \det((A-\lambda I))=0$
	\\Certain ODEs/systems can be solved by eigendecomposition:
	\\Assume soln has form $u(x) = ce^{iwt}$
	\\$\sum F_i=m_i\ddot{x}_i=-k_i(x_i-x_{x-1})-k_{i+1}(x_i-x_{i+1})$
	\\Can use this to write out matrix as \\$A\vec{x}=-\omega^2\vec{x}=\vec{\ddot{x}}$
	\\$\vec{x}(t) = \sum_j C_{\pm j}\vec{w}_j e^{\pm i\omega_j t}$
	\\For this, two eigenvectors then give four solns, accounting for $\pm$ in 
	\\constants and in exponent
	\\for example, eigenvalues are $-\omega^2$
	\\\textbf{Eigendecomposition}
	\\$A\vec{w}=\lambda\vec{w}$
	\\let $\vec{w} = P\vec{w}^*$
	\\so $AP\vec{w}^*=\lambda P\vec{w}^*$,
	\\$A^*=\lambda\vec{w}^*$, $A^*=P^{-1}AP$
	\\Need to find P st $A^*$ diagonal.
	\\Steps: Find eigendecomp P (like Jacobi method)
	\\Find $A^*$, eigenvals are diagonals of $A^*$
	\\eigenvecs are columns of P
	
}
\\
\minibox[frame]{\textbf{Iterative Solvers}
	\\\textbf{(all of my written x's are vectors)}
	\\given A and b
	\\Direct solvers: exact solution if no liminations for precision
	\\(finite processes)
	\\Iterative solvers:
	\\not exact answer, but we have idea for how good
	\\\textbf{Iterative improvement}
	\\$x_0$ is true soln: $Ax_0=b$, $x$ is guess
	\\$Ax=b+r$ r is residual, $x_0=x-\delta x$
	\\$Ax=A(x_0+\delta x)=b+r$, $r=A\delta x$, $\delta x = A^{-1} r$
	\\$x_{i+1} = x_i - \delta x$
	\\\textbf{Jacobi Method}
	\\$Ax=b \qquad (L+D+U)x=b$
	\\$L$ is elements below diag NOT TRIANGULAR; $U$ are above, $D$ is diag.
	\\$x_{n+1} = - D^{-1}(L+U)x_n+D^{-1}b$
	\\Convergence: for multiple dimensions, FP method needs to 
	\\be a contraction mapping to converge ($x_{n+1} = G(x_n)$)
	\\Converges to $x_0$ if all eigvals of J of G are <1 in abs val
	\\similar to 1D: $\abs{g^\prime(x)} < 1$
	\\Easier (but not always necessary): converges if A is row diagonal dominant
	\\Gauss-Seidel: $x_{n+1} = -(L+D)^{-1}Ux_n+(L+D)^{-1}b$
	\\still comes from $A = L + D + U$
	\\\textbf{Increasing Basin of Convergence via Relaxation}
	\\Rewrite Jacobi as weighted sum:
	\\$x_{n+1} = \omega(-D^{-1}(L+U)x_n+D^{-1}b)+(1-\omega)x_n$
	\\only convergent where $0 < \omega < 2$
	\\Under-relaxed if $0 <\omega 1$: interpolated b/w previos soln to new soln
	\\increases basin, calms down oscillatory iterations
	\\Over-relaxed if $1 < \omega 2$: extrapolates from previous soln to beyond new soln
	\\Can therefore give faster conv, but negatively affect basin
	\\to derive these, get two diff x's and then make one you xn, one xn+1
	
	
	
}
}
\newpage
\section*{Integration}
\minibox[frame]{\textbf{Numerical Integration}
	\\\textbf{Why?}: most integrals can't be solved analytically
	\\Often just want numbers, not analytic solns
	\\especially true for high dimensions
	\\automation and speed often more important than analyticity
	\\\textbf{Rectangle Method:}
	\\mesh points range from 0 to $n$, $h$ is mesh spacing
	\\$I=\sum\limits_{i=0}^{n-1}f(x_i)(x_{i+1}-x_i) = h\sum\limits_{i=0}^{n-1}f(x_i)$
	\\\textbf{Midpoint Method:}
	\\$I = h\sum\limits_{i=0}^{n-1}f\left([x_i+x_{i+1}]/2\right)$
	\\\textbf{Trapezoidal Method:}
	\\$I = h\sum\limits_{i=0}^{n-1}f\left([f(x_i)+f(x_{i+1})]/2\right)$
	\\\textbf{SIMPLIFIED Trapezoidal Method:}
	\\$I = \frac{h}{2}(f_0+f_n)+h\sum\limits_{i=1}^{n-1}f_i$
	\\\textbf{Simpson's Methods:}
	\\Use polynomials to approximate the function in small interval
	\\$f(x) \simeq p(x) = \alpha + \beta(x-x_i)+\gamma(x-x_i)(x-x_{i+1})$
	\\$\alpha = f_i, \quad \beta = (f_{i+1}-f_i)/h, \quad \gamma = (f_i-2f_{i+1}+f_{i+2})/2h^2$
	\\$I_i=\int\limits_{x_i}^{x_{i+2}}p(x)\diff x$
	\\$I_i = \frac{h}{3}(f_i+4f_{i+1}+f_{i+2})$
	\\So, \textbf{Simpson's 1/3 Method:}
	\\$I=\frac{h}{3}\left[f_0+4\sum\limits_{i=1,3,5\dots}^{n-1}f_i+2\sum\limits_{i=2,4,6\dots}^{n-2}f_i+f_n\right]$
	\\\textbf{Simpson's 3/8 (cubic) Method:}
	\\$f(x) \simeq p(x) = \alpha + \beta(x-x_i)+\gamma(x-x_i)(x-x_{i+1})+\delta(x-x_i)(x-x_{i+1})(x-x_{i+2})$
	\\gives $I_i=\frac{3h}{8}(f_i+3f_{i+1}+3f_{i+2}+f_{i+3})$
	\\$I=\frac{3h}{8}\left[f_0+3\sum\limits_{i\%3 =1}^{n-1}f_i+3\sum\limits_{i\%3=2}^{n-2}f_i+2\sum\limits_{i\%3=0}^{n-3}f_i+f_n\right]$
	\\\textbf{Summary:} in order of rectangular/midpoint, trap., 1/3, and 3/8:
	\\interpolation is nearest neighbor, linear, quad, cubic; 
	\\no. of points needed to define each $I_i$: 1, 2, 3, 4.
	\\
	\\\textbf{Improper Integrals}
	\\Have upper or lower limit of $\infty$
	\\Has singularity at or b/w upper and lower lims
	\\Cannot be evaluated at or b/w upper and lower ims
	\\Is otherwise integrable (convergent)
	\\Limit at singularity: ex:
	\\$\int\limits_0^2\frac{1}{x^p}\diff x$
	\\converges if $p<1$
	\\Limit at infinity, not over singularity: ex:
	\\$\int\limits_2^\infty\frac{1}{x^p}\diff x$
	\\converges if $p>1$
	\\\textbf{How to evaluate:}
	\\Strategy 1: (semi-infinite interval (a or b inf)) substitute $y=1/x$
	\\Strategy 2:(two inf limits) break into two integrals, then use strat 1
	\\Strategy 3: (power law singularity at end of interval) 
	\\substitute $y=(x-a)^{1-\gamma}, 0\leq\gamma<1$
	\\
	
	


}
\minibox[frame]{
	\\\textbf{Gaussian Quadrature}
	\\Limitations of previous numerical integration methods:
	\\Formula: $\int\limits_a^bf(x)\diff x\simeq \sum\limits_{i=0}^nA_if(x_i)$
	\\$x_i$ are abscissas, $A_i$ are weights, $w(x)$ is weighting function
	\\Underlying assumption that function can be expanded in a basis of polynomials $p_j(x)$
	\\$\int\limits_a^bf(x)\diff x =\sum\limits_{j=0}^{2n+1}c_j\int\limits_a^bp_j(x)\diff x$
	\\EXACTLY reproduced by $\sum\limits_{i=0}^nA_if(x_i)$
	\\The size of missing higher-order terms governed by quality of polynomial expansion
	\\\textbf{More generally:} $\int\limits_a^bF(x)\diff x = \int\limits_a^b w(x) f(x) \diff x$
	\\Weighting function can be singular; $f(x)$ is polynomial-like
	\\$f(x)$ can be decomposed into a set of orthogonal polynomials
	\\Gaussian quadrature is exact if $f(x)$ is a lin comb of 
	\\polynomials $p_j(x)$, $j_{max}\leq 2n+1$, that are \textbf{orthogonal across the weighting function $w(x)$}:
	\\$\int\limits_a^bw(x)p_m(x)p_n(x)\diff x = 0, \quad m\neq n$
	\\\textbf{Absicissas are the $n+1$ roots of } $p_{n+1}(x)=0$
	\\
	\\\textbf{STEPS}
	\\1) Choose $w(x)$ and $f(x)$ based off of original integrand $F(x)$:
	\\$\int\limits_a^bF(x)\diff x = \int\limits_a^b w(x) f(x) \diff x$
	\\2) Choose an orthogonal basis set $p_j(x)$ across $w(x)$ 
	\\As needed, modify oringla integral to put into form assumed by basis set
	\\3) Choose the order of quadrature $n$ then determine weights $A_i$ and abscissas $x_i$:
	\\A) look up
	\\B) Use a generating function?
	\\C) Use principles that abscissas are zeros of $p_{n+1}(x)$
	\\4) Compute integral:
	\\$\int\limits_a^bw(x)f(x)\diff x \simeq \sum\limits_{i=0}^nA_if(x_i)$
	\\Gauss Chebyshev: $A_i=\pi/(n+1)$, $x_i=\cos(\frac{(2i+1)\pi}{2n+2})$
}
\newpage
\section*{Fourier Stuff and ODEs}
\minibox[frame]{
\\Dirac Delta 
\\\textbf{Continuous FT}
\\Decomposiion of some $f(t)$ into $e^{i\omega t}$ basis set
\\$F(\omega)=\int\limits_{-\infty}^\infty f(t) e^{i\omega t} \diff t$
\\$f(t)=\frac{1}{2\pi}\int\limits_{-\infty}^\infty F(\omega) e^{-i\omega t} \diff t$
\\\textbf{Discrete Fourier Transform (DFT)}
\\Define finite interval [-L,L]
\\Define basis set as cos and sine functions that have int no. of oscillations over the interval
\\Discrete Fourier decomp:
\\$f(x) = \frac{a_0}{2} + \sum\limits_{k=1}^\infty (a_k\cos(\frac{k \pi x}{L})  + b_k\sin(\frac{k \pi x}{L}))$
\\$a_k = \frac{1}{L}\int\limits_{-L}^Lf(x)\cos(\frac{k \pi x}{L})\diff x \qquad b_k = \frac{1}{L}\int\limits_{-L}^Lf(x)\sin(\frac{k \pi x}{L})\diff x \qquad a_0 = \frac{1}{2L}\int\limits_{-L}^Lf(x)\diff x$
\\\textbf{From continuous to discrete}
\\$H_n=\Delta\sum\limits_{k=0}^{N-1}h_ke^{2i\pi kn/N}$
\\\textbf{Aliasing and Nyquist Frequency}
\\How fine should our mesh be to capture a function properly?
\\In the discretized form, each freq amplitude is contaminated with other frequencies
\\The result cannot be undone from the discrete values alone.
}
\minibox[frame]{\minibox[frame]{}}
	\\\minibox[frame]{\textbf{ODEs}
		\\\textbf{Rewriting nth order ODE as coupled 1st order}
		\\Method 1: facorization ex:
		\\$y^{\prime\prime}-2y^\prime-3y=f(x) \, \Rightarrow \, [D-3][D+1]y=f(x)$
		\\Let $y_1=y$, $y_2=[D+1]y$
		\\So $y_2 = [D+1]y_1=y_1^\prime + y_1$, $[D-3]y_2=f(x) = y_2^\prime-3y_2$
		\\Method 2: Brute Force
		\\Let $y_1=y$, $y_2=y^\prime$
		\\\textbf{IVP fixed step integrators}
		\\Given $y^\prime (x) = f(x,y)$ and $y(x_0) = y_0$
		\\Compute $y$ on a mesh $y_i=y(x_i)$ with step size $h$
		\\\textbf{Euler's Method}
		\\Use truncated Taylor expansion $y(x+h) = y(x) + y^\prime (x)h + O(h^2)$
		\\$y(x+h) = y(x) + f(x,y(x))h + O(h^2)$
		\\Error accumulation over $n$ steps: $O(h)$.
		\\Continued $\rightarrow$
		
	}
}
\newpage
\minibox[frame]{
}
\minibox[frame]{
	\textbf{Using Fourier Transforms to solve ODEs}
	\\Ex: DDO $m\ddot{x}+\gamma\dot{x}+kx=f(t)$
	\\If $f(t)=Ae^{i\omega_0 t}$, $x(t) =c_1e^{i\omega_0 t}$, $c_1=\frac{A}{-m\omega_0^2+i\gamma\omega_0+k}$
	\\General soln: $X(\omega)=\frac{F(\omega)}{-m\omega^2+i\gamma\omega + k}$
	\\Next, consider the orthogonal basis set of the Fourier decomp: $v_j=e^{i\omega_j t}$
	\\$\dv{}{t}v_j=i\omega_j v_j \quad \rightarrow \quad Av_j=\lambda_jv_j$
	\\\textbf{Fourier basis are eigenfunctions of the differential operator}
	\\So, any system $A\vec{x}=\vec{b}$ can be written as
	\\$x=\sum\limits_jx_jv_j \quad b = \sum\limits_jb_jv_j \quad A\sum\limits_jx_jv_j=\sum\limits_jb\lambda_jx_jv_j=\sum\limits_jb_jv_j$	
	\\Now you have a set of independent equations: $\sum\limits_jx_j=\sum\limits_jb\frac{b_j}{\lambda_j}$
	\\Can be generalized to any linear system with known eigenfuncs of operator
	\\\textbf{Stiff Differential Equations and Implicit Methods}
	\\\textbf{ODE Eigenval Method} Ex: Consider system
	\\$u^\prime = 998u+1998v \quad u(0)=1$
	\\$v^\prime = -999u-1999v \quad v(0)=0$
	\\$\begin{bmatrix}u^\prime \\ v^\prime	\end{bmatrix}=D\begin{bmatrix} u \\ v\end{bmatrix} =\lambda \begin{bmatrix}u\\ v\end{bmatrix} \quad \Rightarrow \quad \begin{bmatrix}998 & 1998\\ -999 & -1999\end{bmatrix}\begin{bmatrix}u\\ v\end{bmatrix}=\lambda\begin{bmatrix}u\\ v\end{bmatrix}$
	\\$\det{A-I\lambda}$ gives eigenvals $\lambda_1, \lambda_2$, eigenvecs $\vec{w}_1, \vec{w}_2$
	\\$\begin{bmatrix}u(x)\\ v(x)\end{bmatrix} = c_1\vec{w}_1e^{\lambda_1x} + c_2\vec{w}_2e^{\lambda_2x}$
	\\\textbf{PROBLEMS:} Matrix ill-conditioned so determinant could result in error. 
	\\Need step size to be $h \leq 2/\lambda$
	\\from $y^\prime=-\lambda y \quad y_{n+1}=y_n+hy^\prime_n=(1-\lambda h)y_n$
	\\Having two totally diff eigenvals mean you can't compute both simultaneously
	\\\textbf{Eigenvals stiff if very different}
	\\\textbf{REMEDY: use Implicit methods}
	\\$y_{n+1}=y_n+hy_{n+1}^\prime=y_n+\lambda hy_{n+1}=\frac{y_n}{1+\lambda h}$
	\\STABLE FOR $\forall h$ especially large $h$, useful for solutions over large time
	\\\textbf{Can be generalized to coupled ODEs}
	\\$\vec{y}^\prime=-C\cdot\vec{y} \quad \vec{y}_{n+1}=(I -Ch)\cdot\vec{y}_n$
	\\$(I-Ch)^n\rightarrow 0$ for $\lambda_1{max} < 1$
	\\$\vec{y}_{n+1}=(I+Ch)^{-1}\vec{y}_n$
	\\\textbf{THIS would be useful for solving that original example.}
	\\Can also write above (to avoid matrix inversion) with 1st order Taylor exp as:
	\\$\vec{y}_{n+1}=\vec{y}_n+h\left[I-h\pdv{\vec{f}(\vec{y}_{n+1})}{\vec{y}}\right]^{-1}\vec{f}(\vec{y}_n)$
	\\This can generalize to higher orders :)
	
	
}
\newpage
\end{document}