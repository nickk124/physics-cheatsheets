\documentclass[10pt]{article}
\usepackage[margin=0.1in]{geometry}
\usepackage{amsmath}
\usepackage{amssymb}
\usepackage{graphicx}
\usepackage{lmodern}
\usepackage{float}
\usepackage{physics}
\usepackage{minibox}
\newcommand*\diff{\mathop{}\!\mathrm{d}}
\begin{document}
	\fontsize{6pt}{7pt}\selectfont
	
\subsubsection{Star Formation, Black Hole Types, Hawking Radiation}
\minibox[frame]{\textbf{Jeans Mass} 
	\\The first criterion for gas cloud collapse:
	\\ $ M > \frac{5R}{G}\frac{k_BT}{\overline{m}}\equiv M_\text{jeans}$
	\\($\overline{m}$ is avg. gas particle mass, $R$ is cloud radius)
	\\$R_\text{jeans}=\left(\frac{15k_bT}{4\pi G \overline{m}\rho}\right)^{1/2}=\left(\frac{3M_\text{jeans}}{4\pi\rho}\right)^{1/3}$
	\\\textbf{Fragmentation}
	\\For some gas cloud above $M_\text{j}$, as cloud collapses,
	\\$\rho \uparrow$, so $M_\text{j} \downarrow$, so now regions of the cloud have $M>M_\text{j}$
	\\This cycle repeats for smaller and smaller subregions,
	\\\textbf{Only if T is constant}
	\\Once $\rho$ increases enough, $l_\text{mfp} \downarrow$: cloud can't cool, then
	\\$T\uparrow$ means $M_j \uparrow$; fragmentation stops
	\\this is one reason why stars form in groups
	\\\minibox[frame]{
		\textbf{Star Birth and Getting on the Main Sequence}
		\\1) Dark cloud cores, then gravitational collapse
		\\2) Clump of gas becomes protostar, then T-Tauri
		\\3a) Pre MS Star: If massive enough, collapses to begin fusion
		\\3b) If not massive enough, stays as a brown dwarf
		\\the more massive the star, the faster it forms
		\\ stars can take 10s of thous to 10s of mil yrs to land on MS}





}
\minibox[frame]{\textbf{Star Clusters}
\\\textbf{Open Clusters}: (stellar nurseries): mostly MS stars, 
\\drift apart as supernovae start popping off	
\\\textbf{Globular Clusters}: Dense, bound for life
\\rare star-star collisions thought to form blue stragglers
\\\textbf{Metallicity:} What the star formed from
\\\minibox[frame]{
	\textbf{Gas Cooling Mechanisms:}
	\\- Dissociation of molecular H, - Ionization of Hydrogen
	\\- Hydrogen recombination (Balmer and lower E $\gamma$ 
	\\can escape (not too high E that would just 
	\\be absorbed by surrounding H)
	\\- Metals: molecular lines, free emission
	\\The more metallic, the better at cooling, the more 
	\\fragmentation, smaller stars form
	\\Diff cooling mechanisms happen at diff temps:
	\\\textbf{From low T to high, in order}:
	\\molecular line emission, low ionization metals, 
	\\H \& He recomb., highly ionized metals, \\free-free (deceleration of a charged particle deflected by 
	\\other charged particle; loses KE as photon)

	
	
	
}
}
\minibox[frame]{\textbf{Types of Black Holes}
		\\\textbf{Stellar Mass}: form by core collapse SN, NS mass transfers 
		\\and mergers ($\sim10 M_\odot$)
		\\\textbf{Supermassive}: formed by merging, and acreetion, found in 
		\\centers of galalxies and grow over cosmic time (mils of $M_\odot$)
		\\\textbf{Intermediate}: 1000's of $M_\odot$, tentative evidence
		\\\textbf{Primordial}: no clear evidence
		\\\minibox[frame]{\textbf{Hawking Radiation}
			\\in empty space, particle-antiparticle pairs
			\\ exist for instant before annihilating into energy
			\\near event horizon, one can fall in, as the other escapes, 
			\\but energy debt must be paid, so BH loses mass.
			\\Therefore, BHs have temps and BB spectrum:
			\\$T = 6.2\times 10^{-8} \text{ K} \times \frac{M_\odot}{M}$
			\\Black holes evaporate if hotter than background CMB $T$: \\($M<0.008M_\text{earth}$)
		
	}
	\\\textbf{IMPORTANT:} $M(<r) = \int_0^r\rho(r')\diff V$, $\diff V = 4\pi r'^2\diff r'$ 
	\\see interstellar gas visibly with:
	\\diffuse cool gas: reflection/scattering
	\\dense cool gas: extinction/reddening (objects appear redder)
	\\hot gas: emits lines (red: H$\alpha$)
	}

\subsubsection{Relativity and Accretion Disks}
\minibox[frame]{\textbf{Einstein's Postulates (1915 (GR))}
\\1) Physics is the same in all ref frames that move at a const vel
\\1b) The speed of light in vacuum is same in all ref frames
\\2) \textbf{A freefall frame in a grav field is indist. from an iner. frame in no grav. field}
\\ 2b) An accelerating frame is indist. from a frame at rest in a grav field with g=a

\\\textbf{Spacetime Diagrams}
\\\minibox[]{
\\\includegraphics[width=0.2\linewidth]{../../Pictures/Screenshots/spacetimed}\\units: s and ls
}
\minibox[frame]{
\minibox[frame]{\textbf{Spacetime Interval}
	\\All observers agree on $\Delta s$
	\\$(\Delta s)^2 = -(c\Delta t)^2 + (\Delta x^2 + \Delta y^2 + \Delta z^2)$ 
	\\\textbf{Proper Time}: $\tau$: time between 2 events \\as seen by obs who sees them happen 
	\\in same place
	\\$(\Delta s)^2 < 0$: timelike
	\\$(\Delta s)^2 = 0$: null, or lightlike
	\\$(\Delta s)^2 > 0$: spacelike
}
\\\minibox[frame]{\textbf{Gravitational Redshift (less rigorous)}
\\$z=\frac{\Delta\lambda}{\lambda_\text{em}}=\frac{-v}{c}=\frac{-g\Delta t}{c}=\frac{\Delta\lambda}{\lambda_\text{em}} = \frac{gh}{c^2}$
\\h is d traveled by $\gamma$
\\according to outside observer
\\redshifted as it exits grav well

}
}
}
\minibox[frame]{\textbf{SC metric:}
	$\diff s^2 = -c^2\diff t^2\left(1-\frac{2GM}{rc^2}\right)+\diff r^2\left(1-\frac{2GM}{rc^2}\right)^{-1}+r^2\diff\theta^2+r^2\sin^2\theta\diff\phi^2$
	\\\textbf{In flat spacetime: }$\diff s^2 = -c^2\diff t^2+\diff r^2+r^2\diff\theta^2+r^2\sin^2\theta\diff\phi^2$
	\\\textbf{Gravitational Time Dilation:} $\diff\tau=\diff t\sqrt{1-\frac{2GM}{rc^2}}$ \\Where $\diff t$ is time seen by somebody infinitely far away.
	\\\textbf{Grav Redshift:} $\lambda_\text{obs}=\lambda_\text{em}\left(1-\frac{2GM}{rc^2}\right)^{-1/2}\quad\frac{\Delta\lambda}{\lambda_\text{em}}=\frac{\sqrt{1-\frac{2GM}{r_2c^2}}}{\sqrt{1-\frac{2GM}{r_1c^2}}}-1$
	\\\textbf{SC Radius} (event hori): $R_s=\frac{2GM}{c^2}$, lowest possible STABLE orbit: $r>\frac{3GM}{c^2}$
	\\\minibox[frame]{\textbf{Accretion Disks}
		\\From MFM, as $M(<r)\simeq M_{BH}$: $v_\text{rot} \simeq \sqrt{GM/r}$
		\\For ring of disk: $T =r^{-3/4}\times \left(\frac{GM\dv{M}{t}}{8\pi\sigma_{SB}}\right)^{1/4}$ Where $M$ is BH Mass
		\\Lotal disk luminosity: $L_\text{tot}\simeq \frac{1}{2}\frac{GM\dv{M}{t}}{r_\text{inner}}$ 
		\\Where $r_\text{inner}$ is the innermost stable orbit ($3R_S$ for BH)
		\\efficiency: how much of M going in is being radiated away: $\eta = \frac{L_\text{tot}}{\dv{M}{t}c^2} = \frac{1}{2}\frac{GM}{r_\text{inner}c^2}$
		\\\textbf{Eddington Luminosity Limit:} $L_\text{tot, max}= \frac{4\pi cGMm_\text{proton}}{\sigma_T}$
		\\$\sigma_T = 6.65\times 10^{-29}\text{ m}^2$, $m_p = 1.673\times10^{-27}$ kg 
		
	}
	
	
}
\subsubsection{Gravitational Lensing and Finding Planets}
\minibox[frame]{\minibox[frame]{\textbf{For top right situation (aligned)}:
		\\$\alpha = \frac{4GM}{c^2b}$ (Where $b$ is dist of closest approach)
		\\$\alpha D_{ls} = \theta D_{os}$
		\\angular disk radius: $\theta_E = \left(\frac{4GM}{c^2}\frac{D_\text{ls}}{D_{ol}D_{os}}\right)^{1/2}$ 
		\\Maybe: $\theta_E = 10^{-4}\text{ as}\left(\frac{M}{M_\odot}\right)^{1/2}\times\left(\frac{D_{ol}}{10\text{ kpc}}\right)^{1/2}$
		\\\textbf{For bottom right situation (nonaligned)}:
		\\$D_{os}\beta+D_{is}\alpha=D_{os}\theta \rightarrow \theta^2-\beta\theta-\theta_E^2=0$
		\\$\theta_{\pm}=\frac{1}{2}\left[\beta\pm(\beta^2+4\theta_E^2)^{1/2}\right]$
		\\Lensing: brightness$\uparrow$; more light focused on us
		\\\textbf{Magnification (rightmost pic)}:
		\\Relative increase in angular size of each image (+ and -):
		\\$a_{\pm} = \frac{\text{img angular size}}{\text{source angular size}} = \frac{\theta_{\pm}\diff\theta_{\pm}}{\beta\diff\beta}=\frac{\theta_{\pm}}{2\beta}\left[1\pm\frac{\beta}{(\beta^2+4\theta_E^2)^{1/2}}\right]$
		\\$a_\text{tot} = a_+ + a_- = \frac{u^2 +2}{u(u^2+4)^{1/2}}$, where $u = \beta/\theta_E$
		\\For small $u$, $a_\text{tot} \simeq 1/u = \theta_E/\beta\qquad \frac{\tau_\text{star}}{\tau_\text{planet}}\simeq\sqrt{\frac{M_\text{star}}{M_\text{planet}}}$
		\\Duration of lensing event: $\tau = \frac{\theta_ED_{ol}}{v}\propto M_\text{planet}$ where $v$ is 
		\\$v$ b/w source and lens (planet) in the plane of the lens
	
		
		
		
	}
	\minibox[frame]{\\\includegraphics[width=0.15\linewidth]{../../Pictures/Screenshots/gravlensing}
	
}
\minibox[frame]{\includegraphics[width=0.15\linewidth]{../../Pictures/Screenshots/gravmagnification}
}
}
\minibox[frame]{
	\textbf{1) Microlensing:} find rockey planets \\w/ long period: one time only
	\\\textbf{2) Radial Velocity of star:} best finding massive plans 
	\\in close orbits (hot jupiters) (didn't form that close)
	\\\includegraphics[width=0.13\linewidth]{"../../Pictures/Camera Roll/WIN_20171118_19_19_36_Pro"}
	\\$M_\text{planet}\sin i\approx \left(\frac{\tau}{2\pi G}\right)^{1/3}\times|v_\text{1, obs}|M_\text{star}$
	\\Where $\tau$ is orbital period, $i$ is shown in pic
	\\\textbf{3) Transit Detection:}$\frac{\Delta f}{f}=\left(\frac{r_\text{planet}}{r_\text{star}}\right)^2$, $P=\frac{r_*+r_p}{d_{(p,*)}}$ 
	\\\textbf{4) Direct Imaging}: 
	\\Only possible for big, young planets, far away from star


}
\subsubsection{Galaxies}
\minibox[frame]{\textbf{General Galaxy Stuff}$\qquad \rho\propto 1/r^2$
	\\From Virial Thm, for ell. gals and gal clusters:
	\\$M = \frac{5\sigma_\text{LOS}^2R}{G}\qquad \sigma_\text{LOS}^2=\sum_N\frac{1}{N}(v_i-\overline{v})^2$
	\\\minibox[frame]{\\\textbf{Active Galactic Nuclei}
		\\Size of the solar system
		\\gas moving relativistically
		\\center has Supermassive Black Hole
		\\Emit mostly x-ray and radio (unlike stars)
		\\Eddington Luminosity applies
		\\AGN Variability: $R<c\Delta t \simeq 37.5$ AU
		\\\textbf{AGN Jets}: extend up to one Mpc
		\\emits at all wavelengths, peaks in IR, UV
		\\\textbf{Types:} 
		\\Seyfert Galaxies: $10^8-10^{11}$ $L_\odot$
		\\Quasars: $3\times 10^{10}-4.3\times 10^{14}$ $L_\odot$
		\\\textbf{Some Non-AGN galaxies have SMBHs} \\(non-accreting) But not all gals has BHs
		\\SMBH mass correlates to central bulge mass \\(but SMBH does not effect gal structure)
		\\$\uparrow$ due to merging galaxies merging SMBHs too
		
	}
}
\minibox[frame]{\textbf{Isolated Galaxy Formation}
\\(Galaxies don't really form in isolation)
\\1) 1st stars form spheroid: Es, S0s 
\\tend to have old stellar populations
\\2) Gas comes in, forms disk as it is 
\\the config of lowest energy that maintains \\$\vec{L}$. New stars form from disk gas
\\\textbf{Galactic Mergers}
\\\textbf{Steps:} 1) random dispersion of stars
\\sphereoids develop (violent relaxation)
\\Dark Matter increases ``target size''
\\Dynamical Friction helps galaxies merge.
\\\textbf{More dyn fric if:} slower relative 
\\motion, greater satellite mass, larger 
\\DM halos (mergers start when halos touch)
\\Gas flows inward due to tidal forces, new 
\\stars form in huge burst due to replenished 
\\gas, now a lot of stars in center
\\\textbf{Most star form. was 10-11 Gyr ago}
\\\textbf{Galaxy Clusters:} filled w hot gas,
\\gravitational lensing implies mass, all
\\indicates lots of DM. mostly elliptical gal
\\because of many mergers
}
\minibox[frame]{\textbf{Classifying Galaxies: Hubble ``Sequence''}
	\\\includegraphics[width=0.37\linewidth]{../../Pictures/Screenshots/hubblecycle}
	\\\textbf{Left to Right:} Gas infall \& Outer disk growth, less spheroid, 
	\\more thin disk, more rotation, more cold gas \& young stars, $\therefore$ bluer
	\\\textbf{Right to Left:}Galaxy interactions (bulge/inner disk growth)
	\\Ellipticals: Red and Dead; Irr: Most common, most gas vs. stars
	\\Spiral: Most star formation (blue light means young stars)
	\\Ellipticals: ``early type'' (misnomer)

}
\minibox[frame]{\includegraphics[width=0.1\linewidth]{../../Pictures/Screenshots/dynamicalfriction}
	
	
	
}
\subsubsection{Cosmology, the Big Bang, Dark Matter, and more Quasars}
\minibox[frame]{
\minibox[frame]{\textbf{More on Quasars}
	\\1) Quasars are SMBH perpetually at 
	\\Edd. Lum. 2) Quasars are not near 
	\\MW b/c they were only common 
	\\during first few Gyr after the BB
	\\because they require a lot of 
	\\gas \& merger activity.
	\\3) Now, gas is mostly used up and 
	\\mergers are less common
	\\4) Not all are strong radio sources
}
\\\minibox[frame]{\textbf{Star Cosmology Players}:
	\\Hubble (extra-gal distances)
	\\Leavitt (Cephied Period-L relation)
	\\Slipher (Redshifts of ``spiral nebulae'')
}
}
\minibox[frame]{\textbf{The Evolving Universe}
	\\Hubble's Law: $v\simeq H_0 d \qquad H_0\simeq 70$ km/(s Mpc)
	\\ $z = \frac{\text{dist today}}{\text{dist at emission}}-1 = \frac{\Delta\lambda}{\lambda}\simeq\frac{v}{c}$ \\\textbf{ABOVE ARE ONLY VIABLE AT SMALL z}
	\\$d=a(t)x$ Where $d$ is actual distance b/w two things, 
	\\$x$ is their constant comoving distance 
	\\$H = \frac{\dot{a}(t)}{a(t)}\qquad H_0 \text{ is evaluated at $t_0$, today; $a(t_0) = 1$}$
	\\So, as $H_0$ is always increasing, so is $\dot{a}(t_0)$. (a-DOT)
	\\\textbf{Cosmol. Redshift:} \\$\frac{c\delta t_{obs}}{c\delta t_{em}}=\frac{a(t_\text{obs})}{a(t_\text{em})}=\frac{\lambda_\text{obs}}{\lambda_\text{em}}=1+z>1$
	\\\textbf{a(t)} is the const. of proportionality b/w comoving
	\\dists \& actual proper dists.
	\\If some $a$ is 5 at some time, that means the object 
	\\was 1/5 the distance it is today back then
	\\most star formation was 10-11 Gyr ago
}
\minibox[frame]{\textbf{The Geometry of the Universe}
	\\Observations \& Cosmo Principle imply the universe is homogenous and isotropic 
	\\(no special positions are directions)
	\\\textbf{FRW Metric:} $\diff s^2 = -c^2\diff t^2 + a^2(t)[\text{geometry of space}]$
	\\$\diff s^2= -c^2\diff t^2 + a^2(t)[\diff x^2 + \diff y^2 + \diff z^2]$ \textbf{ Where dx, dy, dz are all comoving dist.}
	\\\textbf{Spherical Coords:} $\diff s^2 = -c^2\diff t^2 + a^2(t)[\diff r^2 + r^2 \sin ^2 \theta \diff \phi^2 + r^2\diff \theta^2]$
	\\\minibox[frame]{\textbf{Possible Universe Geometries}
		\\$\rho_\text{crit}$ is the total energy density needed for the universe to be flat
		\\\textbf{Closed (Spherical):} $\rho_\text{tot}>\rho_\text{crit}$, parallel lines cross, triangle angles $>180^\circ$, 
		\\will collapse: big crunch (this could be altered by DE though)
		\\\textbf{Open (Hyperbolic):} $\rho_\text{tot} < \rho_\text{crit}$, parallel lines diverge, triangle angle $<180^\circ$,
		\\will expand forerver (not altered by DE)
		\\\textbf{Flat: }$\rho_\text{tot}=\rho_\text{crit}$, parallel lines stay parallel, triangle angles = $180^\circ$,
		\\will expand forever, but at slower and slower rate (altered by DE)
		\\Evidence points to this geometry
		\\Today: $\rho_\text{matter} = 0.3\rho_\text{crit}$
		
	}
	
}
\minibox[frame]{\textbf{The Friedmann Equation (In a flat universe)}
	\\$\boxed{\left(\frac{\dot{a}}{a}\right)^2\equiv H^2=\frac{8\pi G}{3}\rho_\text{tot}}$, $\rho_\text{tot}\equiv\rho_\text{crit}$ (either/both 
	\\are not fixed with time, evaluated at a chosen $H$)
	\\$\rho_\text{crit,0}$ is simply evaluated at today ($t_0$,) $H_0$. 
	\\$\rho_\text{crit,0} = 9\times 10^{-27} kg/m^3$ $\Omega_i \equiv \frac{\rho_\text{i,0}}{\rho_\text{crit}}$ 
	\\for $i \in \{\text{matter, radiation, dark energy} (\Lambda) \}$ 
	\\So $\Omega_i$ is ALWAYS evaluated at today
	\\today, $\Omega_\text{m} \simeq 0.3, \Omega_\text{de} \simeq 0.7, \Omega_\text{r} \simeq 9\times 10^{-5}$
	\\\textbf{Second Friedmann Equation}: (double dot a in numer.)
	\\$\frac{\ddot{a}}{a}=-\frac{4\pi G}{3}\left(\rho_\text{tot}+\frac{3P}{c^2}\right)$ Where $P$ is pressure
	\\$\rho_\text{m} \propto \frac{1}{a^3}$, $\rho_\text{r}\propto T^4$, $T\propto \frac{1}{a}\propto(1+z)$
	\\\textbf{Density and More}
	\\General Case: $\rho \propto a^{-3(1+\omega)}$, $\omega_m=0,\omega_r=\frac{1}{3},\omega_{de}=-1$
	\\$\omega$ is the eqn. of state parameter: $\omega = \frac{P}{c^2\rho}$
	\\From this, we find: $\rho_\text{de}\simeq$ constant, 
	\\$\rho_m=\rho_{(m,0)}a^{-3}\rightarrow\rho_m\propto a^{-3}$
	\\$\rho_r=\rho_{(r,0)}a^{-4}\rightarrow \rho_r\propto a^{-4}$
	\\From homework, we found: \\$H(t)=H_0\sqrt{\Omega_ma(t)^{-3}+\Omega_ra(t)^{-4}+\Omega_\text{de}}$
	\\\textbf{Calculating Age of Universe}:
	\\$t_0=\int\limits_0^{a(t_0)}\frac{\diff a'}{\dot{a}'}=\int\limits_0^1\frac{\diff a'}{a'H(a')}$ 
	\\For radiation only universe: $H=H_0a^{-2}$
	\\\textbf{Calculating Size of (Observable) Universe:}
	\\(derived from FRW metric for some $\gamma$ traveling to now)
	\\radius=$c\int\limits_{t_{em}}^{t_0}\frac{\diff t'}{a}=c\int\limits_{a_{em}}^1\frac{\diff a'}{a'^2H(a')}$
	\\ Where ' just indicates dummy variables
}
\minibox[frame]{
\minibox[frame]{\textbf{The Origins of Matter:}
	\\$\gamma$'s $\leftrightarrow$ $e^+, e^-$ if $E_\gamma=h\nu>E_{e^+/e^-}=mc^2$
	\\ $e^+ and e^-$ continuously annihilate
	\\Equilibrium stops when radiation too cold 
	\\but annihilation contiues, and fast
	\\\textbf{Freeze-out:} annihilations stop b/c the 
	\\partics. are dilute enough; b/c so few partics.\\ left, and b/c the expans. of the universe
	\\\textbf{Matter-Antimatter Asymmetry}
	\\Most likely because of some unexplained \\matter-antimatter asymmetry (like 1,000,000 \\vs 1,000,001)
	\\\textbf{Dark Matter Origins:}
	\\The freeze-out process predict the observed
	\\DM density if DM particles interact via Weak F.
	\\\textbf{IF} DM was produced by freeze-out,
	\\(only possible if it has antiparticles,) then
	\\there should be equal amts of DM and 
	\\DantiM If both DM and M freeze out, 
	\\they won't necessarily have same density \\today, as freezeout is dependent 
	\\on MFP of particles annihilating
}
\\\minibox[frame]{\textbf{Big Bang Nucleosynthesis}
	\\NOT full atoms, just nuclei
	\\$T = 10^9$ K$\qquad 2p^+ + 2n^0 \rightarrow 2D^2\rightarrow \text{He}^4$
	\\$t_0=10$ s, lasted about 200 s
	\\Universe was hot and dense enough to 
	\\trigger this fusion, but cool enough:
	\\$\overline{E}_\gamma=k_bT=0.08$ MeV, $E_\text{deuterium}=-2.22 eV$
	\\Universe had to be cool enough for so not \\even one in a billion photons (due to \\freeze-out) would have enough energy to \\destroy deuterium D
	\\That fusion only possible with free neutrons, \\which decay quickly, so doesn't last long; 
	\\this is why no elements heavier than Be
	\\formed; window was too short before too cool
	\\observations of primordial He and D
	\\show that universe is only 4\% atoms
}
}
\minibox[frame]{
	\minibox[frame]{\textbf{The Cosmic Microwave Background}
		\\He, $p^+$, $e^-$ became atoms
		\\\textbf{Surface of Last Scatter:} 380,000 yrs after BB,
		\\when T = 3000 K, hydrogen atoms formed,
		\\after which photons could travel freely:
		\\$l_{mfp}=\left(\frac{n_{e}}{n_{e\text{ free}}}\right)\times\frac{7\times10^{28}\text{ m}}{(1+z)^3}$
		\\Because more of the free electrons joined nuclei 
		\\(low enough energy free $\gamma$ to not ionize atoms)
		\\so now universe was transparent from then on
		\\$l_{universe}\approx\frac{c}{H}$
		\\Our motion actually redshifts \& blueshifts the CMB
		\\CMB is the perfect BB radiator, T=2.725
		\\confirms hot big bang. there are very tine mK flucs.
		\\for in this
	}
	\\\textbf{Dark Matter Candidates}
	\\\textbf{WIMPS} are leading candidate; leading \\candidate for WIMP: SUSY fermion partners of 
	\\Weak force carriers (partners of W, 
	\\Z, Higgs bosons). Other proposals 
	\\struggle to explain DM's current density
	\\\textbf{MACHOS}: They cant be DM (at least \\more than 10\% of it). Looking to 
	\\the DM halo, MACHOS would produce 
	\\much more gravitational microlensing 
	\\than is actually seen, if it was all MACHOS.
	\\\textbf{Neutrinos:} ruled out because they're \\too hot: the large scale structure that \\neutrionis would make is not what is observed.
	\\\textbf{Ultra-cold H2 globules:} DM can't be made
	\\out of regular matter
	\\\textbf{MOND:} Can explain some things, 
	\\but not everything.
	\\Still continues to not die.
	\\$1^\circ$ disks in CMB, casually separated by 
	\\photon path since BB; yet, still all have
	\\almost exactly same temp
} 
\subsubsection{Misc Info, Derivations}
\minibox[frame]{\textbf{Misc Info}
\\\textbf{HII Regions:} when O, B stars form, they emit radiation 
\\that can ionize surrounding gas, making HII region
\\$\rho/n=\overline{m}\qquad g_{surf}\simeq\frac{GM}{R^2}$
\\Press. broad comparing 2 stars: $\frac{\Delta \lambda_1}{\Delta \lambda_2}=\frac{n_1}{n_2}=\frac{P_1/kT_1}{P_2/kT_2}$
\\$N=M/\overline{m}$, Metallicity is frac. that's not H or He
}
\minibox[frame]{\textbf{Roche Lobe, Tidal:}
	\\$\frac{M_1}{(a-R_L)^2}-\frac{M_2}{R_L^2}=\frac{M_1}{a^2}-\frac{R_L(M_1+M_2)}{a^3}$
	\\$R_L\simeq a\left(\frac{M_2}{3M_1}\right)^{1/3}\qquad F_\text{tide}\simeq\frac{2GMmh}{r^3}$
\\\textbf{Jean's Mass Deriv}:
	\\From V thm: collapse if $U>2K$: $\frac{3GM^2}{5R}>2\times\frac{3Mk_bT}{2\overline{m}}$
\\\textbf{Accretion Disk Temp}:
\\For falling $\diff m$: $\Delta U_\text{grav}=\frac{-GM\diff m}{r^2}\diff r$
\\From MFM: $\Delta KE=\Delta E_{thermal}$
\\$\frac{1}{2}\diff m\left(\sqrt{\frac{GM}{r}}\right)^2-\frac{1}{2}\diff m\left(\sqrt{\frac{GM}{r+\diff r}}\right)^2=\frac{1}{2}\frac{GM\diff m}{r^2}\diff r$
\\$\frac{\Delta E_\text{thermal}}{\diff t}=\frac{1}{2}\frac{GM\diff m \diff r}{r^2 \diff t}$
\\$\frac{\Delta E_\text{thermal}}{\diff t}=L_\text{ring}=\sigma_{SB}T^4(2\pi r\diff r)\times 2$
\\Solve these for each other to get answer.
\\\textbf{Total Disk Luminosity:}
\\$L_\text{tot}=\int_{r_\text{inner}}^{r_\text{outer}}2(2\pi r \diff r)\sigma_{SB}T^4$
\\\textbf{Eddington Luminosity:}
\\Consider some $e^-$ trying to fall into object.
\\There's pt where rad. press overcomes grav. attrac on $e^-$
\\$n_\gamma=\frac{L_\nu}{4\pi r^2 c h \nu}$
\\Scattering rate of $e^-$ = $n_\gamma\sigma_Tc$ Each scatter: $\Delta p_e=\Delta p_\gamma = \frac{h\nu}{c}$
\\$F_\gamma =\dv{p}{t}$, $F_{\gamma_{tot}}=\frac{L_{tot}\sigma_T}{2\pi r^2 c}$; $F_\gamma>\frac{GMm_p}{r^2}$
\\($m_p$ because electrons bound to protons)


	
	
	
	
}
\minibox[frame]{\textbf{Cosmo Redshift Deriv:}
	\\For photon going towards Earth (FRW Metric):
	\\$c^2\diff t^2 = a^2 \diff r^2\rightarrow \diff r =\pm c\frac{\diff t}{a}$
	\\$\int_{r_{gal}}^{r=0}\diff r = \pm c\int_{t_{em}}^{t_{obs}}\frac{\diff t}{a}= c \int_{t_{em}+\delta t_{em}}^{t_{obs}+\delta t_{obs}}$
	\\Double taylor expand on $t_{em}$ and $t_{obs}$
	\\$r_{gal} = c\int_{t_{em}}^{t_{obs}}\frac{\diff t}{a} + \left(\frac{-c}{a}\right)_{t_{em}}\delta t_{em} + \left(\frac{c}{a}\right)_{t_{obs}}\delta t_{obs}$
	\\$0 = \frac{-c}{a_{em}}\delta t_{em} + \frac{c}{a_{obs}}\delta t_{obs}$
	\\\textbf{densities:}
	\\$\dv{\rho}{t}=\frac{3}{8\pi G}\left(2\frac{\ddot{a}}{a^2}-2\frac{\dot{a}^2}{a^3}\right)=-3H(\rho+\frac{P}{c^2})=-3H(1+\omega)\rho$
	
	
}
\end{document}