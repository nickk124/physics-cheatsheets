\documentclass[10pt]{article}
\usepackage[margin=0.1in]{geometry}
\usepackage{amsmath}
\usepackage{amssymb}
\usepackage{graphicx}
\usepackage{lmodern}
\usepackage{float}
\usepackage{physics}
\usepackage{minibox}
\newcommand*\diff{\mathop{}\!\mathrm{d}}
\begin{document}
	\fontsize{6pt}{7pt}\selectfont
\subsubsection{Geometry and Basics}
\minibox[frame]{\includegraphics[width=0.13\linewidth]{angle}}
\minibox[frame]{arc length $l=\theta R$ \textbf{IN RADIANS}
\\$v_t=\omega r, \quad \omega = 2\pi f, \quad F_c = m\omega^2r$
\\ 1 as = $4.85 \times 10^{-6}\,$ rad
\\ $ s = 2R\sin\frac{\theta}{2}\simeq R \theta$ at $ R \gg l$ 
\\ parallax angle $\alpha,\quad \sin\alpha=R/d$
\\R = 1 AU = $1.5 \times 10^8$ km
\\$\alpha \simeq R/d$ as $d \ll R$ \textbf{IN RADIANS}
\\ 1 pc = d such that $\alpha$= 1 as
\\$pc \propto \frac{1}{as}$}
\minibox[frame]{\includegraphics[width=0.24\linewidth]{parallax}}	
\minibox[frame]{Center of Mass: $x_{COM}=(\sum\limits_i M_ix_i)/M_{total}$
\\Where x=0 can be defined however.
\\Solid Angle: $\Omega \equiv A/R^2$ where A, R some\\ surface area outlined by $\Omega$, R is r of sphere.
\\$SA_{sphere} = 4\pi R^2 \quad \diff\Omega=\sin\theta\diff\theta\diff\phi$
\\$V_{sphere} =\frac{4}{3}\pi r^3$
\\OBAFGKM in dec. order of $T_{surf}$.
\\1 dyne = $10^{-5}$ N
\\1 erg = $10^{-7} J$
\\Luminosity classes: I through V, on HR diagram, lines 
\\from top downwards, sorted by width of spectral lines.
}
\subsubsection{Gravitation and Kepler's Laws}
\minibox[frame]{
	\textbf{Shell Theorem:}
	\\1. No grav force on objects inside \textbf{spherical} shell
	\\2. Spherically symmetric objects can be treated as point masses
	\\ \textbf{Kepler's Laws}
	\\1st The C.O.M. of a system does not accel.; 
	\\can choose a ref frame where it's stationary.
	\\3rd: For m orbiting M, with orbital radius R: \\If $M\gg m$: $ P^2 =\frac{4 \pi^2}{GM}R^3$
	\\If M is \textbf{not} $\gg m$: $P^2 =\frac{4 \pi^2}{G(M+m)}a^3$; $MR = mr$; $a= r+R$
	\\$R_{M}=ma/(m+M)\quad r_{m}=Ma/(M+m)$
	\\where $a$ is the \textit{distance between the two masses} and $P$ is the \\orbital period of both.
	\\$F=\frac{GMm}{r^2}$ \\Motions Find Mass: $ v_t^2 = \frac{GM(<r)}{r}$
	\\Sun BB curve lower on side of lower WL cause these \\get absorbed by n=2 H gas in sun atmos.

}
\minibox[frame]{\includegraphics[width=0.3\linewidth]{flux}
	\\Eventually derives: $F_\lambda = \pi I_\lambda \frac{R^2}{d^2}$ (unresolved)}
\minibox[frame]{\includegraphics[width=0.33\linewidth]{stel}
	\\}
\subsubsection{Intensity, Flux, Luminosity, Etc (Light Stuff)}
\minibox[frame]{$I_\lambda=energy/(\diff t \diff A_\perp \diff \Omega \diff \lambda)$
\\$\diff A_\perp$ is the area vector for a chosen ray.
\\$F=energy/(\diff t \diff A \diff \lambda)$ 
\\is the integral of I over all solid angles.
\\$L = energy/\diff t$
\\$I$ and $F$ don't have to be per $\lambda$.
\\$F_{obs\, (unresolved)}=L/4\pi d^2=(R^2F_{surf}/d^2)$
\\$F_{\lambda (resolved)} = \pi I_\lambda  \theta_{pixel}^2$
\\$F_{surf} = \pi I$
\\$L = 4\pi R_{object}^2 F_{surf}$
\\SB Law: $F_{surface} = \sigma T^4$
\\$Period_{variable\, star} \propto L$
\\\textbf{Blackbody}: $B_\lambda=(2hc^2/\lambda^5)/(e^{hc/\lambda kT}-1)$
\\$B_\nu = (2h\nu^3/c^2)/(e^{h\nu/kT}-1)$
\\\textbf{$\therefore E$ is linear wrt to $\nu,$ but not $\lambda$. }
\\Wein's Law: $\lambda_{max} = (2.9 \,mm)/T$ (max of BB curve)
\\\textbf{Doppler (Angular other side)}: $\lambda_{obs} =\lambda_{em}\sqrt{\frac{1-v/c}{1+v/c}}$ 
\\$\Delta \lambda/\lambda = \gamma -1$
\\ If $v \ll c$: $\lambda_{obs} =\lambda_{em}(1-v/c)$
\\$\frac{\lambda_{obs}-\lambda_{em}}{\lambda_{em}}=\frac{-v}{c}=z$ , sign of v is neg if moving away
\\Rot Broadening: $\Delta \lambda/\lambda=2 v_{rot}/c$
\\Thermal Broadening:  $\frac{\Delta \lambda}{\lambda}\approx\frac{2}{c}\sqrt{\frac{2kT}{m_{particle}}}$
\\Pressure Broadening: $\frac{\Delta \lambda}{\lambda}\propto P$
\\local thermodynamic equil (local blackbody:) $l_{mfp} \ll h_T$
\\$h_T$ is the temperature scale height

}
\minibox[frame]{
\minibox[frame]{\includegraphics[width=0.4\linewidth]{resolved}
	\\So for resolved ($>1$ pixel): $F_\lambda = \pi 
	I_\lambda  \theta_{pixel}^2$}
\minibox[frame]{\textbf{Eclipsing Binaries}
\\1. Went through files to determine highest 
\\stdev in Mag (time originally in MJD)
\\2. Supersmoother metric was made to measure 
\\the deviation of adjacent phased-up points:
\\3. Looped through possible periods, phased up 
\\epochs for each, plotted ssm data (average of 
\\abs. dif b/w adjacent, "zeroed" mags \\(mags-median[mags]) vs possible periods to 
\\determine period
\\4. Period used to phase epochs correctly 
\\and plot light curve (zeroed mags vs. 
\\phased epochs)
\\Binary systems: redshift reveals P and v, 
\\that leads to masses.

}
	\\\textbf{0.0.4 Microscopic Physics}
\minibox[frame]{$E_n=\frac{-13.6 \, eV}{n^2} \qquad E_\gamma = hf \qquad f =c/\lambda \qquad \Delta p_x \Delta x > h \quad $
	\\$\Delta n$ requires exact amount of E (like ladder) besides ionization.
	\\$P_{degen}=\kappa\rho^{m+1}$, m =2/3 for non-rel, m=1/3 for rel.}
\\
\minibox[frame]{\textbf{Hydrogen Emission}\\
	\begin{tabular}{|c|c|}
		\hline 
		$n_f$ & Name, Type \\ 
		\hline 
		1 & UV, Lyman \\ 
		\hline 
		2 & Visible, Balmer \\ 
		\hline 
		3 & IR, Paschen \\ 
		\hline 
\end{tabular}}
\minibox[frame]{\textbf{MFP stuff}
%\\$V_{swept out} = l_{path}[\pi (2r)^2]=l_{path}\sigma$
\\cross section $\sigma =\frac{ No.\, of\, events}{(No.\, density\, of\, targets) * (l\, traveled)}$
\\Prob of event $=\sigma l_{path} n$
\\$l_{mfp} = 1/(n\sigma)$
\\$\tau=\frac{l_{medium}}{l_{mfp}}$ , $\tau \gg 1$: optically thick

}
\minibox[frame]{$\int\frac{\diff I_\lambda(l)}{I_\lambda}=-\int n \sigma_\lambda \diff l \quad \tau_\lambda = nl\sigma_\lambda$
\\$\downarrow$
\\$I_\lambda = I_{(\lambda, 0)}e^{-\tau_\lambda}$
}
\\
\\$ \Delta M = M_f - M_i= -2.5\log_{10}(F_1/F_2) \quad \gamma = 1/\sqrt{1-v^2/c^2} \quad t' = t\gamma \quad X_{element}=\rho_{element}/\rho_{total}$ (Mass fraction)
\\hyperfine hydrogen n=1: $f_{up\rightarrow down}= 1420 $ MHz, $\lambda$ = 21 cm}
\subsubsection{Stellar Physics}
\minibox[frame]{\textbf{Virial Thm Stuff and More}
	\\$U_{grav}+2K=0$
\\$K_{system}=\frac{3}{2}NkT=\frac{3}{2}\frac{M_{system}}{\overline{m}_{particle}kT}$
\\$U_{grav} = -\int\limits_0^R\frac{GM(<r)\rho(r)4\pi r^2 dr}{r}$ 
\\\textbf{U is negative!}, increases (less negative) as dist. increases
\\If $\rho$ is constant, $U_{grav}=-\frac{3GM_{star}^2}{5R_{star}}$
\\$E_{total}=U_{grav}/2=\frac{-GM^2}{2R}$ (also neg)
\\$E_{thermal} = -E_{total}$
\\$PV=NkT \quad P=nkT \quad PV = \frac{2}{3}K$ Relativ: $PV=\frac{1}{3}K$
\\$P=-\frac{1}{3}\frac{U_{grav}}{V}\quad |U_{grav}|\approx \frac{GM^2}{R}\quad P \approx \left(\frac{4\pi}{3^4}\right)^{1/3}GM^{2/3}\rho^{4/3}$
\\$P_{radiational}=\frac{1}{3}aT^4 \quad N = M_{total}/\overline{m}$
\\Isotropic Pressure: both radiational and kinetic
\\$L \propto T^4R^2$
\\$L \propto T_{surf}^8$
\\$\uparrow T \,\rightarrow \,\uparrow R$ due to increased P and V

}
\minibox[frame]{\textbf{Stellar Interior ODEs}
	\\$\dv{M(<r)}{r}=4\pi r^2\rho$
\\ $\boxed{\dv{P}{r} = -\frac{GM(<r)}{r^2}\rho}$
\\ $P = n_{(no. density)}kT = \frac{\rho}{\overline{m}_{particle}}kT$
\\ $\dv{L}{r} = 4\pi \rho r^2 \epsilon, \quad \epsilon\equiv \frac{E\, produced}{mass}$
\\
\\ $\dv{T}{r} = -\frac{1}{4\sigma_{S-B}T^3}\cdot\frac{1}{l_{mfp-\gamma}}\cdot\frac{L}{4\pi r^2}$
\\Approximating these as power laws gives:
\\$P \approx \frac{M\rho}{r} \quad M \approx r^3\rho \quad L\approx \frac{T^4r}{\kappa \rho}$
}
\minibox[frame]{\textbf{Thermostat Cycle in core:}
\\fusion rate $\propto T \rightarrow\, T \uparrow \rightarrow\, P \uparrow \rightarrow\,$ core expands
$\rightarrow\,$
\\ core cools $\rightarrow\,$ T/fus rate $\downarrow\, \rightarrow\, P\downarrow \rightarrow\,$ core contracts
$\rightarrow\,$\\$ T\uparrow:$ REPEAT. 
	
\\
\minibox[frame]{\textbf{Stellar Reactions}
	\\p-p chain: $6p^+ 2e^- \rightarrow 2p^+ + ^4He + 2\nu_e$
	\\ $\Delta m =0.7\% \,of\, 4m_p,\quad E = 25.71 MeV$
	\\ For $M>1.2M_\odot$: (also a bit for lesser masses)
	\\CNO Cycle: same net cycle as p-p, just diff. steps
}
}
\\
\minibox[frame]{\textbf{Useful TPS}
	
	
	
}
\subsubsection{Stellar Evolution A: Main Sequence, White Dwarfs, Supermassive Stars}
\minibox[frame]{
\minibox[frame]{\textbf{Main Sequence}
		\\$L \approx M^{3.5}$
		\\MS lifetime $\tau \propto 1/M^{2.5}$
		\\$ L \propto T_{surf}^8$
	
}
\minibox[frame]{\textbf{Massive Stars}
	In chronological order of
	\\ fusion reactant shells in deep core
	\\ (if hot/massive enough):
	\\H, He, C, O, Ne, Si, Fe
	\\ \textbf{end in Core Collapse Supernovae},
		\\(type II, or type Ib/Ic if no H left)
}
\\
\minibox[frame]{\textbf{White Dwarfs}
		\\$ R \propto R_{c}, \quad \overline{\rho} \propto \rho_{c}, \quad L \propto R^2T^4, \quad R_{WD} \approx 7000 \, km \left(\frac{M}{M_\odot}\right)^{-1/3}\propto \frac{1}{m_e}$

\\\textbf{Non-rel}: $M \propto \rho_{c}^{1/2}\rightarrow R_{WD} \propto M^{-1/3}$
\\$\downarrow$ as M added
\\\textbf{Rel:} $M \propto \rho_{c}^{0} \rightarrow$trying to increase $\rho_{c}$ will not increase M
\\Max $M_{WD} = M_{ch} = 1.4 M_\odot$
\\Using Virial Thm, and formula for $P_{degen}$ \& $\rho^{4/3} = \frac{M^{4/3}}{V^{4/3}}$:
\\$\overline{P}V=\left(\frac{3}{8\pi}\right)^{1/3}\frac{hc}{4m_p^{4/3}}\left(\frac{Z}{A}\right)^{4/3}\frac{M^{4/3}}{V^{1/3}} = -\frac{1}{3}U_{grav}=-\frac{1}{3}\frac{GM^2}{R}$
\\\textbf{Where Z is proton count, A is nucleon count}
\\Solve for M, gives: $M = \left[\frac{3}{10}\left(\frac{9}{32\pi^2}\right)^{1/3}\frac{hc}{Gm_p^{4/3}}\right]^{3/2} = 0.8 M_\odot$
\\This is off, actually $M_{ch} = 1.4 M_\odot$, minimum mass for Ia novae
\\
\\Once WD gets enough mass via accretion, carbon in core begins\\ fusing, except not, thermostat is broken, as unlike normal stars, T \\does not affect P. 
\\Surface grav: $g=\frac{GM}{R^2}$
}
}
\minibox[frame]{\includegraphics[width=0.5\linewidth]{evolution}
\\Stars peel of main sequence like a banana, starting from highest L}
\subsubsection{Stellar Evolution B: Core Collapse Supernovae, Neutron Stars}
\minibox[frame]{\textbf{Core Collapse Timeline}
	\\\textbf{FOR STARS} M$>8M_\odot$
	\\\textbf{1.} iron core grows until reaching $M_{ch}$, 
	\\after which $e^-$ degen pressure can't support it anymore.
	\\\textbf{2.} Core collapses, hot $\gamma$'s break up iron \\nuclei into He, then p and n, cooling the core \\(these rxns use E, at the end basically only p, n left):
	\\$ \gamma + ^{56}Fe \rightarrow 13^4He +4n$
	\\then $ \gamma + ^4He \rightarrow 2p + 2n$
	\\\textbf{3.} $e^-$ absorbed into p (and other nuclei)
	\\making $\nu_e$ \& n, further cooling core:
	\\$e^- + p \rightarrow n + \nu_e$
	\\\textbf{4} Now the core is mostly neutrons.
	
}
\minibox[frame]{
\minibox[frame]{\\\textbf{Energy released in core collapse}
	\\Total change in $U_{grav} = 5 \times 10^{53}\, ergs$
	\\Energy used in:
	\\10\% for nuclei breakup
	\\1\% for debris KE
	\\0.01\% for luminosity
	\\89\% in released $\nu$'s
}
\\
\minibox[frame]{\textbf{Other Stellar Interior Equations}
	\\Where L is luminosity and dL 
	\\is luminosity of shell, $\epsilon(r)$ \\is energy produced per unit mass:
	\\$L+\diff L = L+M_{shell} \times \epsilon(r)$
	\\$ \diff L = 4\pi \rho r^2 \diff r \times \epsilon(r)$
	\\$\boxed{\dv{L}{r}=4\pi \rho r^2 \epsilon (r)}$}
}
\minibox[frame]{\textbf{Neutron Stars} 
	\\(the collapsed iron cores of supergiants)
	\\$R_{NS}\simeq \frac{m_e}{m_n}\simeq 11\, km \left(\frac{M}{1.4M_\odot}\right)^{-1/3}$
	\\$M_{max} \approx 0.2\left(\frac{Z}{A}\right)^2\left(\frac{hc}{Gm_p^2}\right)^{3/2}\times m_p$
	\\ Where $\frac{Z}{A} = 1$ (no protons)
	\\So $M_\text{max} \approx 5.6M_\odot$ \textbf{WRONG!}
	\\Accounting for strong force b/w 
	\\neutrons, and degen pressure 
	\\contributing to gravitation (due to GR):
	\\$\boxed{M_{NS, max} \simeq 2-3 M_\odot}$
	\\Above this, a black hole will form,
	\\ as degen press. can't beat gravity
	\\\textbf{Pulsars}
	\\Jets created by strong magnetic field
	
}
\minibox[frame]{
\minibox[frame]{\textbf{Degeneracy Pressure}
	\\Using the stellar interior equations, \\and $P_{degen}=\kappa\rho^{m+1}$, you arrive at:
	\\$ \dv{P}{r}=\kappa (m+1)\rho^m\dv{\rho}{r}$
	\\To solve this, switch to dimensionless 
	\\variables, $x=\rho/\rho_{center}$
	\\$ R \propto R_{core}; \overline{\rho} \propto \rho_{core}$}
\\
\minibox[frame]{\textbf{Wien's Law Derivation}
\\set $\dv{B_\lambda}{\lambda} = 0=\frac{xe^x}{e^x-1} -5$
\\Where $x=hc/\lambda k T$
\\Solved numerically to find $x\simeq 4.965$

}
}
\subsubsection{Derivations}
\minibox[frame]{\textbf{Shell Thm}
	\\1. Divide spherical shell of mass $M$, radius $R$ into individual rings 
	\\with masses $\diff M$. $a$ is dist from test mass to given ring. 
	\\Use $F_g$ formula to determine$\diff F_g$ b/w test mass $m$ and $\diff M$.	
	\\2. $\diff M$ can be writtin in terms of sphere surface \\density $\rho$. Use this to write $\diff F_g$ in terms of \\integratable differentials. Integrate over all rings ($\theta$) to find $F_g$.
	\\$ \diff F_g = \frac{Gm\diff M}{a^2}\cos\phi, \quad \diff M = \rho 2 \pi R\sin\theta R \diff\theta$
	\\\includegraphics[width=0.2\linewidth]{shell}
	\\3. Law of cosines used to simplify integral. $F_g$ is found to be 0.
	
}
\minibox[frame]{
\minibox[frame]{$B_\lambda$ from $B_\nu$
	\\Using chain rule, $B_\lambda = B_\nu \cdot \pdv{\nu}{\lambda}$
	
}
\\
\minibox[frame]{\textbf{Stefan Boltzmann}
	\\$F_{surf} = \pi \int_0^\infty B_\nu \diff \nu$, use $u(\nu)=\frac{h\nu}{kT}$
}
\\
\minibox[frame]{\textbf{Sun core temp}
	\\Find P(0) from Hydro equil eqn
	\\Use $PV=NkT=\frac{M_{core}}{\overline{m}}$
	\\$T\approx \frac{P_{center}\overline{m}}{\rho_{sun}k}$ 
}
\\
\minibox[frame]{
	\\\textbf{Radiative Transport:} \\Consider a shell with thickness dr. 
	\\$f_{shell_{net}}=f_{up}-f_{down}$
	\\$f_{shell_{net}}=\sigma_{SB}[T^4(r)-T^4(r+\diff r)]$
	\\Simplified with Taylor expansion:
	\\$ \frac{L(r)}{4\pi r^2}=\sigma_{SB}[T^4(r) - [T^4 + \dv{T^4}{r}\diff r]]$
	\\We are talking about how much energy \\a photon can carry (radiative \\transport) therefore $\diff r \equiv l_{mfp-\gamma}$
	\\$\boxed{\dv{T}{r} = -\frac{1}{4\sigma_{S-B}T^3}\cdot\frac{1}{l_{mfp-\gamma}}\cdot\frac{L}{4\pi r^2}}$
}
}
\minibox[frame]{\textbf{Angular (and regular) Doppler}
	\\A star with constant $\vec{v}$ releases wave A at $t_1$, then B at $t_2$
	\\$c_\Delta t$ is, at $t_2$, dist b/w A and where star was at $t_1$.
	\\$l_{ab}$ is dist b/w A and B at $t_2$
	\\\includegraphics[width=0.18\linewidth]{../../Downloads/20171004_214728}
	\includegraphics[width=0.18\linewidth]{../../Downloads/20171004_214740}
	\\At small $\Delta t, l_{ab}\approx c\Delta t -v\Delta t\cos\theta$
	\\$\lambda_{obs} = \frac{l_{ab}}{n} \quad n$ (no. of peaks and troughs b/w waves)$ =\frac{c\Delta t}{\lambda_{em}\gamma} $
	\\$\rightarrow \boxed{\lambda_{obs} = \lambda_{em}\gamma(1-\frac{v}{c}\cos\theta)}$
	
	
}
\\
\minibox[frame]{\textbf{Virial Thm}
	\\
	S = $\sum_{\alpha}\vec{p}_\alpha \cdot \vec{r}_\alpha \quad \left< \frac{dS}{dt}\right> \equiv \frac{1}{N \, measurements} \sum_{n=0}^{N-1} \frac{dS}{dt}=$
	\\$\boxed{\frac{1}{T}\int_0^T \frac{dS}{dt} dt= 0 \,as\, T\rightarrow \infty}$
	$T$: total time of all measurements.
	\\ $\left< \frac{dS}{dt} \right> = 0 = \left< \sum_\alpha \frac{d\vec{p}_\alpha}{dt}\cdot \vec{r}_\alpha + \vec{p}_\alpha \cdot\frac{d\vec{r}_\alpha}{dt} \right>= \left< \sum_\alpha \vec{F}_\alpha \cdot \vec{r}_\alpha \right> + 2\left<K_{total}\right>
	$
	\\$ \left< \frac{dS}{dt} \right> = 0 = \left< \sum_\alpha\left(\sum_{i\neq \alpha} \vec{F}_{i \rightarrow \alpha} \cdot \vec{r}_\alpha\right) \right> + 2\left<K_{total}\right>
	$
	\\The inner summation term can be divided up into $\alpha > i$ and \\$\alpha < i$ sums. These are just mirrored, $\therefore$
\\	$ \left< \frac{dS}{dt} \right> = 0 = \left< \sum_\alpha\left(\sum_{i<\alpha} \vec{F}_{\alpha,i} \cdot (\vec{r}_\alpha-\vec{r}_i)\right) \right> + 2\left<K_{total}\right>
	$
	\\
	$ \left< \frac{dS}{dt} \right> = 0 = \left< \sum_\alpha U_{i\alpha} \right> + 2\left<K_{total}\right>
	\rightarrow \boxed{\left<U\right> + 2\left<K_{total}\right> =0}
	$
}
\minibox[frame]{
\minibox[frame]{\textbf{Hydrostatic Equilibrium}
\\ For cylinder slice of star w/ thickness dr, mass dm,
\\$F_g=\frac{G\diff m M(<r)}{r^2} $
\\$F_{up} = AP(r) \quad F_{down} = AP(r-dr)$
\\$F_{down}+F_g=F_{up}\rightarrow A(P(r)-P(r+\diff r)) = \frac{G(\rho(r) A \diff r) M(<r)}{r^2}$
\\Cancel out A's, simplify left side to become H.E. eqn.
}
\\
\minibox[frame]{\textbf{Hydrostatic Eq. to Virial Thm}
\\Multiply both sides of HS eqn by r,\\then integrate from 0 to R wrt r, \\right side is then $U_{grav}$.
\\Left side becomes $-3\overline{P}V$ (integrating 
\\P(r) over star V), so $U_{grav}=-3\overline{P}V$.
\\Using $PV=\frac{2}{3}K$, $U_{grav} +2K = 0$: Virial Thm
}
}
\end{document}