\documentclass[10pt]{article}
\usepackage[margin=0.1in]{geometry}
\usepackage{amsmath}
\usepackage{amssymb}
\usepackage{graphicx}
\usepackage{lmodern}
\usepackage{float}
\usepackage{physics}
\usepackage{minibox}
\newcommand*\diff{\mathop{}\!\mathrm{d}}
\begin{document}
	\fontsize{6pt}{7pt}\selectfont
\subsubsection{Cosmology, Part 2}
\minibox[frame]{comoving distance: $r = c\int_{a_{em}}^1\frac{\diff a}{a^2 H(a)}\qquad\qquad T\propto \frac{1}{a}\qquad \qquad H\equiv \text{expansion rate}$
\\comoving horizon:$r = c\int_{0}^{a_{em}}\frac{\diff a}{a^2 H(a)}\qquad\qquad \ddot{a}>0: \text{acceleration}$ 
\\\minibox[frame]{\textbf{Mysteries and how Inflation Solves Them}
\\\textbf{Horizon problem:} $\theta \simeq\frac{r_{hor}}{r_{cmb}}\simeq 1^\circ$
\\\textbf{Flatness Problem:} How did we manage to happen to get this super precise, flat universe?
\\The geometry of the universe determines angular size of the fluctuations
\\today, $\Omega \equiv 1.005\pm0.007$, so $\sim$2 min after BB, $\abs{\Omega-1}<10^{-16}$
\\{Another Problem:} there isnt enough matter to make universe flat at that $\Omega$
\\\textbf{Initial Fluctuations Mystery:} where did the initial $T$ fluctuations come from
\\In a decel. univ, we can see more and more of the comoving universe over time
\\In accel, we can see less and less over time
\\\textbf{An accel Univ becomes more and more flat} (decel becomes more curved)
\\\textbf{Inflation makes a tiny path of early univ and stretches it to cover all observable univ}
\\\textbf{Fixes Initial Fluctuations Problem}: $\rho$ not UNIFORM during inflation,
\\quantum flucs lead to hot \& cold regions
\\in a decel univ, quantum flucs pop in and out of existence
\\during inflation, quantum flucs are stretched outside the horizon, and are then ``frozen''
}
}
\minibox[frame]{\textbf{Expansion}
\\The content of the universe determines how it expands. From Friedmann Eqn:
\\$\dot{\rho} + 3H\left[\rho + \frac{P}{c^2}\right]=0 \qquad w  \equiv \frac{P}{c^2\rho} \qquad \Rightarrow \rho = \rho_0a^{-3(1+w)}$
\\Another form of Fried 2nd eqn: $\ddot{a}/a = -\frac{4\pi G}{3}(1+3w)\rho$
\\radiation:$a(t) \propto t^{1/2}\quad$ matter: $a(t)\propto t^{2/3}$
\\If Universe is matter-dom, expans decels; so larger apparent vels at a certain dist
\\compared to non-decel univ
\\If DE-dom, expan accels, so apparent vels smaller (exp used to be slower)
\\\textbf{Constant expansion not same as const H!}
\\from v vs. d graph (showing accel and decel,) $H_0$ is slope.
\\no evidence about whether universe will ever stop expanding
\\\minibox[frame]{\textbf{Inflation} (A SPECIFIC TIME OF HIGH EXPANSION)
	\\Volume of universe suddenly went up by $\sim 10^{30}$
	\\$a(t)\propto e^{Ht}$, $H$ basically constant during inflation
	\\\textbf{The Inflaton}: the scalar field that drove inflation
	\\has a scalar potential $V(\phi)$. Inflaton's energy density nearly constant during inflation
	\\Inflation occurs when $V(\phi)$ is flat, giving nearly const. energy density
	\\$V(\phi)$ is NOT $\propto$ energy density
	\\total energy density was const during inflation, then going down during rad dom then 
	\\matter dom, now const again with DE dom.
	\\Density much higher than today during inflation. 
	\\Density of matter, rad greatly decreased during 
	\\\textbf{Gravitational Waves from Inflation}
	\\CMB temp flucs: $\frac{\Delta T_{cmb}}{T_{cmb}}\sim\frac{V(\phi)}{\text{inflaton velocity}}$
	\\the quantum flucs during inflation left imprints on spacetime; these grav waves
	\\leave imprint on CMB polarization, can use to measure $V(\phi)$
	
	
}
}
\subsubsection{Dark Matter and Energy (And Cosmic Distances)}
\minibox[frame]{Problem: at CMB time, normal matter tighly coupled to photons
	\\CMB temp flucs show that photons and normal matter had tiny density flucs at CMB time
	\\by z=10, these dens flucs would grow by about a factor of 100 (to $\Delta \rho /\rho \simeq 100 \times 10^{-5}$
	\\This is too small to form any structure, yet we see gals at z>7, and simulations show
	\\structure forms around z$\sim$20 to match obs
	\\\textbf{The CMB need Dark Matter}: dueling forces of grav and press prevent photons
	\\and normal matter from collapsing prior to recombination
	\\press and grav lead to acoustic oscill. in primordial plasma
	\\to explain structure formation and CMB, we need pressureless matter
	\\\textbf{Dark Matter particles must be heavy, neutral, and stable prior to recomb}
	\\\minibox[frame]{\textbf{Dark Energy}
		\\CMB demands flatness. If mostly matter, it would only be 9.3 Gyr old
		\\Basic info: pressure$=w\times$density$\times c^2$
		\\$w_r=1/3$, $w_m$=0, $w_a<-1/3$
		\\\textbf{What is it? Option 1: A cosmological constant} $\Lambda$:
		\\$w=-1$ (constant density); interpret as vacuum energy:
		\\QFT predicts $\rho_\Lambda\simeq M_\text{Pl}^4$, but observed: $\rho\simeq 10^{-120} M_\text{Pl}^4$ Why so small?
		\\\textbf{Option 2: Quintessence (a scalar field):}
		\\when you need cosmic acceleration, invent a scalar field
		\\like inflaton, need a slowly varying scalar field w near constant $V(\phi)$
		\\Very diff E scales: $10^7$ eV$\lesssim E_\text{infl}\lesssim 10^{25}$ eV
		\\$E_\text{DE}\simeq$ 0.002 eV; quintessence is dynamical: $w \gtrsim -1$
		\\\textbf{Option 3: Change Gravity}:
		\\change gravity: $f(H)=\frac{8\pi G}{3}(\rho_m+\rho_r)$ (mat, rad only)
		\\we have to change grav b/w gals w/o changing it in solar system
		\\time delay and lensing around Sun confirm GR
		\\GR is sensitive. solution chameleon grav, higher dim grav, massive grav...
	}


}
\minibox[frame]{\textbf{Cosmic Distances}
	\\\textbf{1)} Comoving distance (see formula upper left)
	\\\textbf{2)} Physical distance $d=ar$ ($=r$ today)
	\\\textbf{3)} Angular Diameter Distance: with known physical size $R_{phys}$, 
	\\angular res of $\theta$, dist $d_A = R_{phys}/\theta$.
\\Now with known comoving size, $R_\text{com}$, $r=R_\text{com} /\theta$.
\\$\theta = \frac{a_{em}R_{com}}{d_A} \quad\Rightarrow\quad d_A=a_{em}r$
\\\textbf{Luminosity Distance}
\\$d_L = r_{com}/a_{em}$; $F_{obs} = \frac{L}{4\pi d^2_L}$
\\Energy per photon changes as universe expands:
\\$L_{em}=\left(\frac{energy}{photon}\right)_{em}\times\left(\frac{photons}{time}\right)_{em}$
\\$F_{obs} = \left(\frac{energy}{photon}\right)_{obs}\times\left(\frac{photons}{time}\right)_{obs} \times \frac{1}{4\pi r^2}$ Where rightmost term is comoving area
\\$\left(\frac{energy}{photon}\right)_{obs} = a_{em}\left(\frac{energy}{photon}\right)_{em}\qquad\Rightarrow\qquad \left(\frac{energy}{photon}\right)_{obs}=\frac{\delta t_{em}}{\delta t_{obs}}\left(\frac{No. of photons}{\delta t_{em}}\right)$
\\$\left(\frac{photons}{time}\right)_{obs}=a_{em}\left(\frac{No. photons}{\delta t_{em}}\right)$
\\Flux Ratio: $F_1/F_2=(d_{2,L}/d_{1,L})^2$
\\\minibox[frame]{\textbf{Cosmic Distance Ladder}
	\\Radar, Parallax, Main seq fitting, cepheid variables, 
	\\white dwarf SN, then finally Hubble's Law
	
}
}
\subsubsection{Misc Important Stuff to know}
\minibox[frame]{ABCDE... sequence: in order of decreasing H line strength. Strongest molecular lines: coolest (M)
\\Galaxy changes: dust inflow creates disk growth, new stars
\\Galactic mergers make more spheroidical
\\\textbf{Galaxy composition:}
\\\includegraphics[width=0.3\linewidth]{../../Pictures/Screenshots/galaxygas}
\\\includegraphics[width=0.3\linewidth]{../../Pictures/Screenshots/neutronstars}

\\\textbf{Great Debate:} Shapley: MW was only galaxy, Sun not at center
\\\textbf{Curtis:} MW was one of many spiral nebulae
\\We calibrate Hubble's Law with Type Ia supernovae
\\$L=\sigma_{SB}T^44\pi R^2$
\\\textbf{Hydrostatic EQ:} pressure acts above and below layer; weight of gas pushes down
\\\texmtbf{Taylor expansion: }$f(x_0+\delta x)\simeq f(x_0)+f^\prime(x_0)\delta_x$
\\\textbf{Relativistic Virial:} $U+K=0$
\\\textbf{fusion is very sensitive to temp}
\\\textbf{Sun: corona} outer most, hot but wispy, below is photosphere, much cooler
\\\textbf{Supernovae:} Ia leaves no remnant, type II leaves NS or BH
\\\textbf{SC Metric:} $r$ is radius of observer
\\\textbf{moon eclipses sun:} sun has emission line spectrum
\\\textbf{shell burning} always continues to happen
\\\textbf{strong force} is responsible for E released in fusion
\\\textbf{MAIN SEQUENCE:} $R\propto M$
\\\textbf{moon spectrum}: BB reflected from Sun
\\\textbf{Sun is G Type}
\\\textbf{Galaxy mergers}: usually evolve to left side of tuning fork
\\$L/L_\odot=(R/R_\odot)^2(T/T_\odot)^4$
\\\textbf{Edwin Hubble used Period Lum relationship to measure d to Andromeda3}
\\\textbf{Pressure Broadening Greater for Low Mass Stars}

}
\\\minibox[frame]{\textbf{Ultimate Fate of the Universe}
	\\If DE is constant (vacuum E, $w=-1$ or decr, (quintessence = scalar field, $w>-1$,
	\\isolated MW and heat death
	\\If DE is increasing ($w<-1$,) big rip ($a\rightarrow\infty$,) occurs in $\sim60$ Gyr,
	\\gals destroyed 120 $Myr$ before that; star systems unbound about $6$ months before 
	\\rocky planets explode hour before end, atoms destroyed $10^{-19}$ s before
	\\Matter dominated: 50K yrs
	\\current observations indicate $w<-1.2$.
}
	\\ \textbf{Intensity:} Energy per unit time, per unit area that the light passes through, per unit solid angle that it passes through after that area, per unit wavelength.
	\\ degeneracy pressure happens when you don't have high enough temp for normal pressure to even be noticable.
	\\ the MFP is not just the average distance traveled, its the furthest the a particle is most likely able to penetrate
	\\ Epoch: time position in period
	\\ warm gas emits light because temperature causes collisions which creates excited atoms
	\\ avg particle mass in a star is 0.5$m_p$ (in core too)
	\\ Neutron degen pressure is similar to electron degen pressure, except their densities can be higher, as higher mass than electrons means much higher momenta before becoming relativistic.
	\\ Pressure is not really directional: think of it as a scalar quantity that, for example, according to the Hydrostatic Equil. Equation, changes over height.
	\\ Magnitude scales inversely with perceived brightness, or flux. We see a star with negative magnitude as brighter than one with positive magnitude.
	\\ $U_{grav}$ is negative by convention! Pay attention to this sign in virial thm calculations and such. This is because positive work must be done to pull the two objects apart from one another. The convention is that at infinite distance, U=0.
	\\ for stars, $E_{thermal} = -E_{total}$., $E_{total}$ is NEGATIVE!! As stars age, energy is lost (radiated away) so $E_tot$ DECREASES!
	\\\textbf{Line Broadening} 
	\begin{enumerate}
		\item \textbf{Rotational Broadening}: The absorption lines in a star will be spread due to varying degrees of redshifting and blueshifting due to the rotation of the star. As such, what would be an individual wavelength absorbed by gases in the stars atmosphere gets red shifted and blueshifted, stretching the gap.
	\end{enumerate}
	as gas temp increases, more energetic (purple) emissions
	\\ total intensity B found from integrating $B_\lambda$ over all wavelengths
	\\ Continuum/Thermal spectrum is BB. Stars are approx BB.
	\\ most reactions involving neutrinos involve the Weak Force.
	\\ as star temp increases, CNO cycle becomes more prevalent, and p-p becomes less so. And does slightly happen in the sun.
	\\ fusion happening is very sensitive to temperature
	\\ TMS is not actually a sequence.
	\\ Lower energy photons come from within deeper in the stars, as they run into less atoms.
	\\ absorption lines in star's spectra are due too certain gases, molecules in the star's atmosphere absorbing certain wavelengths. Molecules not in hotter stars because the molecular bonds can't form for very long, as other high KE particles will knowck apart.
	\\ OBAFGKM sequence is out of order because it was originally ordered ABCDE... in order of decreasing hydrogen line strength
	\\ stars are mostly Hydrogen (~70\% hydrogen by mass)
	\\ Degeneracy Pressure: due to Pauli and Heisenberg, the higher the density (cramming fermions into tight spaces) the higher their momentum, the higher the pressure.
\end{document}